%!TEX root = Slides.tex
\part{Datenbankprogrammierung}
\label{part:programmierung}

\section{Systemarchitekturen}
\subsection{Grundlagen}

\begin{frame}[t]
\frametitle{\insertsection}
\framesubtitle{\insertsubsection}
\structure{Benutzer und Anwendungen wollen auf Daten einer Datenbank zugreifen}\\[4pt]
\begin{itemize}
 \item Abfragedetails bzw. Abfragesprachen wie SQL sollen nach Möglichkeit vor Anwender verborgen bleiben
 \item Datenhaltung soll austauschbar sein -- Umzug auf anderes (R)DBMS soll minimale bis keine Auswirkungen auf Anwendung haben
 \item (R)DBMS soll zentrale Schnittstelle f\"ur Datenzugriff bilden
\end{itemize}
\abs
Im Laufe der Zeit haben sich mehrere Architekturmodelle gebildet.
\end{frame}

\begin{frame}[t]
\frametitle{\insertsection}
\framesubtitle{\insertsubsection}	
\structure{2-Schichten-Architekturen (2-Tier Architectures)}
\begin{figure}
	\begin{tikzpicture}[
 		block/.style={draw, fill=white,	rectangle, minimum width={width("Magnetometer2222")+2pt},	font=\small}]
			\node[block, fill=blue!30, align=center] (tier1) {Präsentations-\\Schicht};
 			\node[block, fill=blue!30, align=center] (tier2) [below=0mm of tier1] {Anwendungs-\\logik};
 			\node[block, fill=red!30, align=center] (tier3) [below=4mm of tier2] {Datenhaltung}; 
 			\draw (tier2) -- (tier3);				
 			\node[block,fill=blue!30, align=center] (tier21) [left=30mm of tier1] {Präsentations-\\Schicht};
 			\node[block,fill=red!30, align=center] (tier22) [below=4mm of tier21] {Anwendungs-\\logik};
 			\node[block,fill=red!30, align=center] (tier23) [below=0mm of tier22] {Datenhaltung}; 
 			\draw (tier21) -- (tier22);
	\end{tikzpicture}
	\caption{\label{fig:2tier}Zwei alternative 2-Schichten-Architekturen}
\end{figure}		
\end{frame}

\begin{frame}[t]
\frametitle{\insertsection}
\framesubtitle{\insertsubsection}		
\onslide
\structure{Eigenschaften klassischer 2-Schichten-Architekturen}\\[4pt]
	\begin{itemize}
		\item Programmiersprachen: häufig 4GL-Sprachen 
		% Fourth generation language oder kurz 4GL: 
		% Programmiersprachen bzw. -umgebungen, die darauf ausgerichtet sind, rasch – mit möglichst wenigen Codezeilen – 
		% für bestimmten Anwendungsbereich Funktionen oder komplette Anwendungen schreiben zu können und
		% Integration von Datenbank- und Benutzerschnittstellenanweisungen in die Programmiersprache
		%    Befehle zur Abfrage und Manipulation einer Datenbank
		%    Befehle zur einfachen Erstellung von Benutzeroberflächen
		% 4GL-Programmiersprachen: Java, C++, auch SQL
		\begin{itemize}
			\item Speziell für die Entwicklung datenzentrischer Anwendungen
			\item Meist automatische Erstellung von Benutzeroberflächen auf Basis des Datenmodells 
		  \item Leichte und schnelle Entwicklung
		\end{itemize}
		\pause
		\item Nachteile
		\begin{itemize}
			\item Starke Kopplung zwischen den einzelnen Schichten 
			\item Keine Komponente kann ersetzt werden, ohne die anderen Komponenten anpassen zu m\"ussen
			\item Große Systeme werden schnell un\"ubersichtlich 
		\end{itemize}
	\end{itemize}
\pause
\abs
\structure{Durch die schwierige Wartbarkeit und mangelnde Flexibilität sind klassische 2-Schichten-Architekturen 
	heute eher die Ausnahme.}\\[4pt]
\pause
\structure{Moderne 2-Schichten-Architekturen haben Eigenschaften von 3-Schichten-Architekturen
	mit API-separierter Anwendungslogik und Datenhaltung in unterer Schicht}				
\end{frame}

\begin{frame}[t]
\frametitle{\insertsection}
\framesubtitle{\insertsubsection}
\structure{3-Schichten-Architekturen}\\[8pt]
\begin{figure}
	\centering
	\begin{tikzpicture}[block/.style={draw,	fill=white,	rectangle, minimum width={width("Magnetometer2222")+2pt},	font=\small}]
 	\node[block, fill=blue!30, align=center] (tier1) {Client Tier};
	\node[block, fill=green!30, align=center] (tier2) [below=4mm of tier1] {Middle Tier};
	\node[block, fill=red!30, align=center] (tier3) [below=4mm of tier2] {Server Tier}; 
	\draw (tier1) -- (tier2); 
	\draw (tier2) -- (tier3);
	\end{tikzpicture}
	\qquad	
	\begin{tikzpicture}[block/.style={draw,	fill=white,	rectangle, minimum width={width("Magnetometer2222")+2pt},	font=\small}]
	\node[align=center] (tier1f) {};
	\node[align=center] (tier2f) [below=4mm of tier1f] {$\Longrightarrow$};
	\node[align=center] (tier3f) [below=4mm of tier2f] {}; 
	\draw (tier1f) -- (tier2f); 
	\draw (tier2f) -- (tier3f);
	\end{tikzpicture}
	\qquad
	\begin{tikzpicture}[block/.style={draw,	fill=white,	rectangle, minimum width={width("Magnetometer2222")+2pt},	font=\small}]
	\node[block, fill=blue!30, align=center] (tier1c) {Pr\"asentation};
	\node[block, fill=green!30, align=center] (tier2c) [below=4mm of tier1c] {Anwendung};
	\node[block, fill=red!30, align=center] (tier3c) [below=4mm of tier2c] {Datenhaltung}; 
	\draw (tier1c) -- (tier2c); 
	\draw (tier2c) -- (tier3c);
	\end{tikzpicture}
	\caption{\label{fig:3tier}3-Schichten-Architektur}
\end{figure}		
\end{frame}

\begin{frame}[t]
\frametitle{\insertsection}
\framesubtitle{\insertsubsection}
\onslide
\structure{Eigenschaften von 3-Schichten-Architekturen}\\[4pt]
\begin{itemize}
	\item Client-Schicht ausschließlich für Pr\"asentation der Ergebnisse und Daten 
	\item Anwendungslogik befindet sich in mittlerer Schicht 
	\begin{itemize}
		\item Berechnungsvorschriften
		\item Abläufe
		\item $\dots$
	\end{itemize}
	\item Datenhaltung in Server-Schicht kein Bestandteil der Anwendungslogik 
	\begin{itemize}
		\item Einheitliche Zugriffskonzepte und Schnittstellen
	\end{itemize}
\end{itemize}
\pause
\abs
\structure{Durch klare Schnittstellen/Zust\"andigkeiten k\"onnen die einzelnen Schichten modifiziert werden, 
	ohne zu starte Einfl\"usse auf benachbarte Schichten bef\"urchten zu m\"ussen.}
\\[4pt]
\pause
\structure{Zur Erinnerung: Moderne 2-Schichten-Architekturen haben Eigenschaften von 3-Schichten-Architekturen
	mit API-separierter Anwendungslogik und Datenhaltung}				
\end{frame}

\subsection{Zugriff auf Datenbanken}
\begin{frame}[t]
\frametitle{\insertsection}
\framesubtitle{\insertsubsection}
\onslide
\structure{Zugriffskonzepte und Schnittstellen f\"ur programmatischen Zugriff auf Datenbanken}\\[4pt]
\begin{description}
	\item[ODBC] Open Database Connectivity. Klassische Schnittstelle für Zugriffe auf RDBMS
	\item[OLEDB] Designierter Nachfolger von ODBC, jedoch komplexer
	\item[DAO] Data Access Objects. In erster Linie von Microsoft entwickelt, kann auch auf ODBC-Quellen zugreifen
	\item[ADO] ActiveX Data Objects. Zugriff auf tabellenartige Datenquellen (RDBMS, CSV etc.)
	\item[ADO.NET] Reimplementierung von ADO auf~.NET-Framework von Microsoft
	\item[JDBC] Java Database Connectivity. F\"ur Java-Programmierung, vergleichbar mit ODBC 
\end{description}
\pause
\abs
\structure{In Vorlesung: JDBC}
\end{frame}

\section{JDBC}
\subsection{Einführung}

\begin{frame}[t]
\frametitle{\insertsection}
\framesubtitle{\insertsubsection}
\onslide
\begin{itemize}
	\item Zuerst von Sun Microsystems (jetzt Oracle) entwickelt
	\item DBMS-neutrale Schnittstelle (API) f\"ur Datenbankzugriffe
	\item Direkte Nutzung von SQL-Statements[4pt]
%	\pause
%	\item Liegt in verschiedenen Versionen vor:
%	\begin{itemize}
%		\item JDBC V1 konnte herstellerunabhängige Verbindungen zu Datenbanken aufbauen und einfache Queries verarbeiten
%		\item JDBC V2 konnte änderbare Ergebnismengen verarbeiten, außerdem kannte es neue Datentypen und Connection Pooling
%		\item JDBC V3 wurde mit weiteren Änderungsmethoden sowie benutzerdefinierten Datentypen erweitert
%		\item JDBC V4 ist in erster Linie ein Performance-Update 
%	\end{itemize}
\end{itemize}
\end{frame}

\begin{frame}[t]
\frametitle{\insertsection}
\framesubtitle{\insertsubsection}
\structure{Idee von JDBC}\\[4pt]
\begin{columns}[c]
	\begin{column}{.35\textwidth}
		\begin{figure}
			\begin{tikzpicture}[block/.style={draw,	fill=white,	rectangle, minimum width={width("Magnetometer2222")+2pt},	font=\small}]
			\node[block, fill=blue!30, align=center] (tier1) {Java\\Application};
			\node[block, fill=blue!20, align=center] (tier2) [below=0mm of tier1] {JDBC -- API};
			\node[block, fill=orange!60, align=center] (tier2b) [below=0mm of tier2] {JDBC -- Impl.};
			\node[drop shadow, fill=orange, cylinder,minimum height=0.5cm, draw, shape border rotate=90, minimum width=2cm] 
			(db) [below=5mm of tier2b] {DB};
			\draw (tier2b) -- (db);
			\end{tikzpicture}
			\caption{\label{fig:jdbc}JDBC-Konzept}
		\end{figure}		
	\end{column}
	\begin{column}{.65\textwidth}
		\begin{itemize}
			\item JDBC-API befindet sich im Package \texttt{java.sql}
			\item Java-Applikation und JDBC in Anwendungsschicht
			\item Kommunikation mit Datenbanken durch (propriet\"are) \textbf{JDBC-Treiber}, die JDBC-API implementieren.
		\end{itemize}
	\end{column}
\end{columns}
\end{frame}

%%%%%%%%%%%%%%%%%%%%%%%%%%%%%%%%
% wird voererst weggelassen
%%%%%%%%%%%%%%%%%%%%%%%%%%%%%%%%
\nowrite{ 
\begin{frame}[t]
\frametitle{\insertsection}
\framesubtitle{\insertsubsection}
\begin{columns}
	\begin{column}{.48\textwidth}
		\begin{figure}
		 \pgfuseimage{jdbctypes}
		 \tiny Bildquelle: Wikipedia
		 \caption{JDBC Treibertypen}
		\end{figure}		
	\end{column}
	\begin{column}{.48\textwidth}
		\begin{itemize}
			\item Typ-1-Treiber sind Bridge-Treiber zwischen Schnittstellentechnologien
			\item Typ-2-Treiber arbeiten direkt proprietären Bibliotheken 
			\item Typ-3-Treiber nutzen einen Middleware-Server
			\item Typ-4-Treiber kommunizieren direkt mit der DB 
		\end{itemize}
		Die meisten JDBC-Treiber sind Typ-4-Treiber.
	\end{column}
\end{columns}
\end{frame}
}
%%%%%%%%%%%%%%%%%%%%%%%%%%%%%%%%

\subsection{API -- Erste Schritte}

\begin{frame}[t]\frametitle{\insertsection}
\framesubtitle{\insertsubsection}
\structure{Die wichtigsten Objekte der JDBC-API}
\begin{figure}\scalebox{1}{
		\begin{tikzpicture}[
		block/.style={draw,	fill=white,	rectangle, minimum width={width("Magnetometer2222")+2pt},	font=\small}]
		\node[block, fill=blue!50, align=center] (rs) {ResultSet};
		\node[block, fill=blue!50, align=center] (st) [below=10mm of rs] {Statement};
		\node[block, fill=blue!50, align=center] (rs2) [right=40mm of rs] {ResultSet};
		\node[block, fill=blue!50, align=center] (pst) [below=10mm of rs2] {PreparedStatement};
		\node[block, fill=blue!50, align=center] (con) [below right=5mm of st] {Connection};
		\node[block, fill=blue!50, align=center] (dm) [below=10mm of con] {DriverManager};
		\draw (rs) -- (st); 
		\draw (rs2) -- (pst); 
		\draw (pst) -- (con);
		\draw (st) -- (con); 
		\draw (con) -- (dm);
		\end{tikzpicture}
	}
\caption{\label{fig:APIStruktur}JDBC API-Struktur}
\end{figure}					
\end{frame}

\begin{frame}[fragile]
\frametitle{\insertsection}
\framesubtitle{\insertsubsection}
\onslide
\structure{\texttt{\textbf{DriverManager}} verwaltet die installierten und geladenen JDBC-Treiber}\\[8pt]
\begin{columns}
	\begin{column}{.3\textwidth}
		\begin{figure}\scalebox{.45}{
				\begin{tikzpicture}[block/.style={draw,	fill=white,	rectangle, minimum width={width("Magnetometer2222")+2pt},	font=\small}]
				\node[block, fill=blue!10, align=center] (rs) {ResultSet};
				\node[block, fill=blue!10, align=center] (st) [below=10mm of rs] {Statement};
				\node[block, fill=blue!10, align=center] (rs2) [right=40mm of rs] {ResultSet};
				\node[block, fill=blue!10, align=center] (pst) [below=10mm of rs2] {PreparedStatement};
				\node[block, fill=blue!10, align=center] (con) [below right=5mm of st] {Connection};
				\node[block, fill=blue!50, align=center] (dm) [below=10mm of con] {DriverManager};
				\draw (rs) -- (st); 
				\draw (rs2) -- (pst); 
				\draw (pst) -- (con);
				\draw (st) -- (con); 
				\draw (con) -- (dm);
				\end{tikzpicture}
			}				
		\end{figure}
	\end{column}
	\begin{column}{.65\textwidth}
		\pause
		Treiber in System/Applikation vorab bekannt machen:
		\begin{enumerate}
			\item JDBC-Treiberpaket (\texttt{jar}) als Teil des \texttt{CLASSPATH} 
			\pause
			\item JDBC-API in Applikation: \texttt{import java.sql.*} 
			\pause
			\item Laden des propriet\"aren JDBC-Treibers (alternativ):
			\begin{itemize}
				\item In Applikation: \texttt{Class.forName("{org}.postgresql.Driver");}
				\item Bei Ausf\"uhrung: \texttt{java~-Djdbc.drivers=org.postgresql.Driver MyApp}
			\end{itemize} 
		\end{enumerate}
		%\lstset{language=Java}
		%\lstset{escapeinside={(*@}{@*)}}
		%\begin{lstlisting}[xleftmargin=3ex,numbers=none]
		%try {
		%Class.forName("com.mysql.jdbc.Driver").newInstance(); (*@\label{lst:classforname}@*)
		%} catch (Exception e) {
		%...
		%}
		%\end{lstlisting}
		%Zeile \ref{lst:classforname} lädt den Treiber unter dem angegebenen Namen, so dass er von der JVM verwendet werden kann.
	\end{column}
\end{columns}
\end{frame}

\begin{frame}[fragile]
\frametitle{\insertsection}
\framesubtitle{\insertsubsection}
\structure{\texttt{\textbf{Connection}}-Objekt repräsentiert Verbindung zur Datenbank}\\[8pt]
\begin{columns}
	\begin{column}{.3\textwidth}
		\begin{figure}\scalebox{.45}{
				\begin{tikzpicture}[block/.style={draw,	fill=white,	rectangle, minimum width={width("Magnetometer2222")+2pt},	font=\small}]
				\node[block, fill=blue!10, align=center] (rs) {ResultSet};
				\node[block, fill=blue!10, align=center] (st) [below=10mm of rs] {Statement};
				\node[block, fill=blue!10, align=center] (rs2) [right=40mm of rs] {ResultSet};
				\node[block, fill=blue!10, align=center] (pst) [below=10mm of rs2] {PreparedStatement};
				\node[block, fill=blue!50, align=center] (con) [below right=5mm of st] {Connection};
				\node[block, fill=blue!10, align=center] (dm) [below=10mm of con] {DriverManager};
				\draw (rs) -- (st); 
				\draw (rs2) -- (pst); 
				\draw (pst) -- (con);
				\draw (st) -- (con); 
				\draw (con) -- (dm);
				\end{tikzpicture}
			}				
		\end{figure}
	\end{column}
	\begin{column}{.65\textwidth}
		Von \texttt{DriverManager} mit Hilfe einer \textit{connection URL} erzeugt:
\lstset{language=Java}
\lstset{escapeinside={(*@}{@*)}}
\begin{lstlisting}[xleftmargin=3ex, numbers=none]
 ...
 String url = "jdbc:postgresql://<host><:port></database>";
 Connection conn = DriverManager.getConnection(url,uname,pwd);
 ...
\end{lstlisting}
	\end{column}
\end{columns}
\end{frame}

\begin{frame}[fragile]
\frametitle{\insertsection}
\framesubtitle{\insertsubsection}
\structure{\texttt{\textbf{Statement}}}\\[8pt]
\begin{columns}
	\begin{column}{.3\textwidth}
		\begin{figure}\scalebox{.45}{
				\begin{tikzpicture}[block/.style={draw,	fill=white,	rectangle, minimum width={width("Magnetometer2222")+2pt},	font=\small}]
				\node[block, fill=blue!10, align=center] (rs) {ResultSet};
				\node[block, fill=blue!50, align=center] (st) [below=10mm of rs] {Statement};
				\node[block, fill=blue!10, align=center] (rs2) [right=40mm of rs] {ResultSet};
				\node[block, fill=blue!10, align=center] (pst) [below=10mm of rs2] {PreparedStatement};
				\node[block, fill=blue!10, align=center] (con) [below right=5mm of st] {Connection};
				\node[block, fill=blue!10, align=center] (dm) [below=10mm of con] {DriverManager};
				\draw (rs) -- (st); 
				\draw (rs2) -- (pst); 
				\draw (pst) -- (con);
				\draw (st) -- (con); 
				\draw (con) -- (dm);
				\end{tikzpicture}
			}				
		\end{figure}
	\end{column}
	\begin{column}{.65\textwidth}
		\begin{itemize}
			\item \texttt{Statement} repr\"asentiert einen SQL-Befehl
			\item Mit \texttt{Connection}-Objekt erzeugt:
		\end{itemize}
\lstset{language=Java}
\lstset{escapeinside={(*@}{@*)}}
\begin{lstlisting}[numbers=none]
...
Statement stmt = conn.createStatement(); 
...
\end{lstlisting}
	 \begin{itemize}
	  \item Ausf\"uhrung des \texttt{Statement}-Objekts f\"uhrt zur Rückgabe eines \texttt{ResultSet} ...
   \end{itemize}
	\end{column}
\end{columns}
\end{frame}

\begin{frame}[fragile]
\frametitle{\insertsection}
\framesubtitle{\insertsubsection}
\structure{\texttt{\textbf{ResultSet}}}\\[8pt]
\begin{columns}
	\begin{column}{.3\textwidth}
		\begin{figure}\scalebox{.45}{
				\begin{tikzpicture}[block/.style={draw,	fill=white,	rectangle, minimum width={width("Magnetometer2222")+2pt},	font=\small}]
				\node[block, fill=blue!50, align=center] (rs) {ResultSet};
				\node[block, fill=blue!10, align=center] (st) [below=10mm of rs] {Statement};
				\node[block, fill=blue!10, align=center] (rs2) [right=40mm of rs] {ResultSet};
				\node[block, fill=blue!10, align=center] (pst) [below=10mm of rs2] {PreparedStatement};
				\node[block, fill=blue!10, align=center] (con) [below right=5mm of st] {Connection};
				\node[block, fill=blue!10, align=center] (dm) [below=10mm of con] {DriverManager};
				\draw (rs) -- (st); 
				\draw (rs2) -- (pst); 
				\draw (pst) -- (con);
				\draw (st) -- (con); 
				\draw (con) -- (dm);
				\end{tikzpicture}
			}				
		\end{figure}
	\end{column}
	\begin{column}{.65\textwidth}
		\begin{itemize}
			\item \texttt{ResultSet} als Ergebnis der Ausf\"uhrung eines \texttt{Statement}-Objekts
			\item \texttt{ResultSet} repr\"asentiert und beinhaltet die Ergebnismenge (Tupeln) eines SQL-Befehls
		\end{itemize}
	\end{column}
\end{columns}
\end{frame}

\begin{frame}[t]
\frametitle{\insertsection}
\framesubtitle{\insertsubsection}
\onslide
\structure{\texttt{\textbf{ResultSet}} und \texttt{\textbf{Statement}}}\\[8pt]
\structure{Unterschiedliche Ausf\"uhrungsmethoden:}
\begin{itemize}
	\item \texttt{Statement.execute(String sql)}
	\begin{itemize}
		\item F\"uhrt SQL-Befehl \texttt{sql} auf Datenbank aus
		\item Rückgabe: \texttt{true}, wenn Ausf\"uhrung erfolgreich, sonst \texttt{false}
		\item Resultate danach mit Methode \texttt{getResult()} abrufen. R\"uckgabe: \texttt{ResultSet}
	\end{itemize}
	\pause 
	\item \texttt{Statement.executeQuery(String sql)}
	\begin{itemize}
		\item F\"uhrt SQL-Befehl \texttt{sql} aus 
		\item R\"uckgabe: \texttt{ResultSet} 
		\item \texttt{ResultSet} ist \texttt{null} im Fehlerfall
	\end{itemize}
	\pause 
	\item \texttt{Statement.executeUpdate(String sql)}
	\begin{itemize}
		\item F\"uhrt SQL-Befehl \texttt{sql} aus
		\item R\"uckgabe: Anzahl (\texttt{int}) der vom Update betroffenen Tupel
	\end{itemize}
\end{itemize}
\end{frame}

\begin{frame}[fragile]
\frametitle{\insertsection}
\framesubtitle{\insertsubsection}
\structure{\texttt{\textbf{ResultSet}} -- Beispiel}\\[4pt]
\lstset{language=Java}
\lstset{escapeinside={(*@}{@*)}}
\begin{lstlisting}[xleftmargin=3ex, numbers=none, showstringspaces=false]
...
Statement stmt = conn.createStatement(); 
String sql = "SELECT * FROM \"MySchema\".\"Film\"";

if(stmt.execute(sql)){
 ResultSet rs = stmt.getResult(); 
 // Resultat mit ResultSet rs verarbeiten - naechste Folie
} 
\end{lstlisting}
\end{frame}

\begin{frame}[fragile]
\frametitle{\insertsection}
\framesubtitle{\insertsubsection}
\onslide
\structure{\texttt{\textbf{ResultSet}}}\\[4pt]
\begin{itemize}
\item \texttt{ResultSet} weist intern eine tabellarische Struktur auf und kann daher \"ahnlich wie eine Relation gesehen werden
\pause
\item F\"ur das Traversieren eines \texttt{ResultSet} werden entsprechende Methoden bereitgestellt
\begin{itemize}
\item \texttt{beforeFirst()} setzt Zugriffs-Pointer \textbf{vor} den ersten Eintrag
\item \texttt{first()} setzt den Zugriffs-Pointer \textbf{auf} den ersten Eintrag
\item \texttt{next()} setzt den Zugriffs-Pointer auf den nächsten Eintrag
\end{itemize}
\item Traversieren eines \texttt{ResultSet} meist mit Hilfe einer Schleife -- n\"achste Folie
\pause
\item Typisierter Zugriff auf Daten eines Tupels des \texttt{ResultSet}:
\begin{itemize}
\item \texttt{getInt(..)}, \texttt{getString(..)},\texttt{getDouble(..)}, ...
\item Zugriff über Spaltenindex bzw.~Spaltenname m\"oglich: z.B. \texttt{getInt(3)}, \texttt{getInt("Jahr")},
\texttt{getString("Name")}
\end{itemize}
\end{itemize}
\end{frame}

\begin{frame}[fragile]
\frametitle{\insertsection}
\framesubtitle{\insertsubsection}
\structure{\texttt{\textbf{ResultSet}} -- Beispiel}
\lstset{language=Java}
\lstset{escapeinside={(*@}{@*)}}
\begin{lstlisting}[xleftmargin=3ex, showstringspaces=false, numbers=none]
...			
ResultSet rs = stmt.getResult(); 

//ResultSet - Zeiger vor ersten Eintrag stellen
rs.beforeFirst(); 

//Durchlaufen des ResultSet
while (rs.next()) {
 System.out.println("Name des Films: " + rs.getString(2)); 
}
...
\end{lstlisting}
\end{frame}

\begin{frame}[fragile]\frametitle{\insertsection}
\framesubtitle{\insertsubsection}
\structure{\texttt{\textbf{ResultSet}} -- Metadaten}\\[4pt]
\begin{itemize}
\item \texttt{ResultSet} enthält Metadaten (Attributnamen, Datentypdefinitionen etc.).
\item Diese Informationen können mit einem Objekt vom Typ \texttt{ResultSetMetaData} abgerufen werden.
\end{itemize}
Beispiel:
\lstset{language=Java}
\lstset{escapeinside={(*@}{@*)}}
\begin{lstlisting}[xleftmargin=3ex, numbers=none]
...
ResultSet rs = stmt.getResult(); 
ResultSetMetaData rsmd = rs.getMetaData(); 

for (int i = 1; i <= rsmd.getColumnCount(); i++){
 String spaltenname = rsmd.getColumnLabel(i);
 String spaltentyp = rsmd.getColumnTypeName(i);
}
\end{lstlisting}
\end{frame}

\begin{frame}[fragile]
\frametitle{\insertsection}
\framesubtitle{\insertsubsection}
\structure{\texttt{\textbf{PreparedStatement}}}\\[8pt]
\begin{columns}
	\begin{column}{.3\textwidth}
		\begin{figure}\scalebox{.45}{
				\begin{tikzpicture}[block/.style={draw,	fill=white,	rectangle, minimum width={width("Magnetometer2222")+2pt},	font=\small}]
				\node[block, fill=blue!10, align=center] (rs) {ResultSet};
				\node[block, fill=blue!10, align=center] (st) [below=10mm of rs] {Statement};
				\node[block, fill=blue!10, align=center] (rs2) [right=40mm of rs] {ResultSet};
				\node[block, fill=blue!50, align=center] (pst) [below=10mm of rs2] {PreparedStatement};
				\node[block, fill=blue!10, align=center] (con) [below right=5mm of st] {Connection};
				\node[block, fill=blue!10, align=center] (dm) [below=10mm of con] {DriverManager};
				\draw (rs) -- (st); 
				\draw (rs2) -- (pst); 
				\draw (pst) -- (con);
				\draw (st) -- (con); 
				\draw (con) -- (dm);
				\end{tikzpicture}
			}				
		\end{figure}
	\end{column}
	\begin{column}{.65\textwidth}
		\begin{itemize}
			\item H\"aufig wiederkehrende Statements können durch ein \texttt{PreparedStatement} vorbereitet werden. 
			\item DBMS nimmt \texttt{PreparedStatement} entgegen
			\begin{itemize}
				\item bereitet den optimierten Ausführungsplan intern vor
				\item verwendet den optimierten Ausführungsplan fortan (im Kontext einer \texttt{Connection})
			\end{itemize}
		\end{itemize}
	\end{column}
\end{columns}
\end{frame}

\begin{frame}[fragile]
\frametitle{\insertsection}
\framesubtitle{\insertsubsection}
\onslide
\structure{\texttt{\textbf{PreparedStatement}} -- Syntax}
\lstset{language=Java}
\lstset{escapeinside={(*@}{@*)}}
\begin{lstlisting}[xleftmargin=3ex, showstringspaces=false]
String sql = "SELECT * FROM \"Film\" WHERE \"Name\" = ? AND \"Jahr\" = ?";

// PreparedStatement in RDBMS im Kontext der Connection conn
PreparedStatement pst = conn.prepareStatement(sql); 

// Parameter fuer PreparedStatement setzen
pst.setString(1,'Tatort'); (*@\label{lst:pst}@*)
pst.setInt(2,2011); (*@\label{lst:pst2}@*)

// SQL mit PrepraredStatement ausfuehren
ResultSet rs = pst.executeQuery(); 
\end{lstlisting}
\pause
\begin{itemize}
	\item Im SQL-String \texttt{sql} werden \texttt{?} als Platzhalter f\"ur Parameter verwendet.
	\item Diese Platzhalter werden mit \texttt{set}-Methoden belegt.
	\begin{itemize}
		\item In Zeile \ref{lst:pst} wird der erste Parameter auf den Wert \texttt{'Tatort'} gesetzt. 
		\item In Zeile \ref{lst:pst2} wird der zweite Parameter auf den Wert \texttt{2011} gesetzt. 
	\end{itemize}
\end{itemize}
\end{frame}

\subsection{Connection Pooling}

\begin{frame}[t]\frametitle{\insertsection}
	\framesubtitle{\insertsubsection}
    \structure{Der Auf- und Abbau von Verbindungen ist ein Flaschenhals bei der Datenbankprogrammierung}

    \begin{itemize}
    	\item Aufbau von Verbindungen kostet im Allgemeinen sehr viel Zeit 
    	\item Auf einer einzelnen Verbindung kann nicht parallel gearbeitet werden, d.h. mehrere Threads müssen 
    	ihre eigenen Verbindungsobjekte erzeugen
    \end{itemize}

    \begin{block}{Connection Pooling}
    \begin{itemize}
    	\item Mehre Verbindungen werden beim Start der Anwendung aufgebaut und in einem Pool gehalten
    	\item Die Verbindungen aus dem Pool können sofort verwendet werden
    	\item Nicht mehr benötigte Verbindungen werden nicht abgebaut, sondern in den Pool zurückgegeben 
    	\item Mit Parametern kann das Verhalten des Pools gesteuert werden (wie viele Verbindungen usw.)
    \end{itemize}
    \end{block}
\end{frame}

\begin{frame}[fragile]\frametitle{\insertsection}
	\framesubtitle{\insertsubsection}
    \structure{Ein Connection Pooling muss nicht mehr selbst implementiert werden}

    Es gibt Bibliotheken, die ein Connection Pooling JDBC-Konform implementieren. Eine dieser Bibliotheken ist 
    die Open Source Bibliothek C3PO.

    \begin{itemize}
    	\item C3PO kann heruntergeladen werden unter \url{http://sourceforge.net/projects/c3p0/}
    	\item Paket muss im \texttt{CLASSPATH} liegen und kann sofort verwendet werden
    	\item Verbindungen zu Datenbanken werden über eine eigene Datenquelle (Data Source) bereitgestellt
    \end{itemize} 
\end{frame}


\begin{frame}[fragile]\frametitle{\insertsection}
	\framesubtitle{\insertsubsection}
    \begin{lstlisting}[xleftmargin=3ex]
//Erstellen einer Datenquelle
ComboPooledDataSource cpds = new ComboPooledDataSource(); 
cpds.setDriverClass("com.mysql.jdbc.Driver");
cpds.setJdbcUrl("jdbc:mysql://localhost/wwidb1");
cpds.setUser("db");
cpds.setPassword("db"); 
cpds.setMinPoolSize(5); 
cpds.setAcquireIncrement(5);
cpds.setMaxPoolSize(100); 

//Verwendung der Datenquelle
Connection con = cpds.getConnection();
...
\end{lstlisting}
\end{frame}

\begin{frame}[fragile]\frametitle{\insertsection}
	\framesubtitle{\insertsubsection}
    \structure{Die meisten Application Server für Web Anwendungen verfügen über eine eigene Connection Pooling Möglichkeit mittels JNDI}
    \vspace{3mm}

    Beispiel Apache Tomcat: \texttt{server.xml} und Java-Code gegenübergestellt
    \begin{columns}

	    \begin{column}{.48\textwidth}
	    	\lstset{language=xml}
			\lstset{escapeinside={(*@}{@*)}}

			\begin{lstlisting}[xleftmargin=3ex, basicstyle=\ttfamily\tiny]
<Context>
 <Resource name="jdbc/TestDB" 
  auth="Container" type="javax.sql.DataSource" 
  maxActive="100" 
  maxIdle="30" 
  maxWait="1000" 
  username="db" 
  password="db" 
  driverClassName="com.mysql.jdbc.Driver"
  url="jdbc:mysql://localhost:3306/wwidb1"/> 
</Context>
			\end{lstlisting}
	    \end{column}
	    \begin{column}{.48\textwidth}
\lstset{language=Java}
			\lstset{escapeinside={(*@}{@*)}}
			\begin{lstlisting}[xleftmargin=3ex, basicstyle=\ttfamily\tiny]
//Initialisierung
Context initContext = new InitialContext(); 
Context envContext = 
(Context)initContext.lookup("java:/comp/env");
DataSource ds = 
 (DataSource)envContext.lookup("jdbc/TestDB");
			
//Jetzt kann eine Verbindung angefordert werden
Connection conn = ds.getConnection(); 
			\end{lstlisting}
	    \end{column}
    \end{columns}
Der Application Server verwaltet einen eigenen Connection Pool und bedient Java-Anwendungen 

\end{frame}

\subsection{Entwicklung von Anwendungen mit Datenbankanbindung}

\begin{frame}
	\frametitle{\insertsection}
	\framesubtitle{\insertsubsection}
	\structure{Wie es \textbf{nicht} geht:}
	\begin{itemize}
		\item Datenbankverbindungen und -abfragen im Code verstreuen
		\item Immer wieder durch ResultSets iterieren und quasi tabellarisch arbeiten
	\end{itemize}
	
	\alert{Entwerfen Sie eine geeignete Softwarearchitektur und nutzen Sie die Möglichkeiten, die Ihnen die Objektorientierung bietet!}
	
\end{frame}

\begin{frame}
	\frametitle{\insertsection}
	\framesubtitle{\insertsubsection}
	\structure{Systematischer Aufbau:}
	\begin{itemize}
		\item Entwurf von Containerklassen für Ihre Entitäten, die \textit{ausschließlich} aus getter- und setter-Methoden bestehen
		\item Entwurf einer Zugriffsschnittstelle, mit der  auf den Containerklassen gearbeitet wird
		\item Implementierung der Schnittstelle für die jeweils konkrete Datenhaltung!
		\item Greifen Sie im Code \textit{ausschließlich über die Schnittstelle} auf die Daten zu, am besten mittels einer Generator- oder Factory-Klasse!
	\end{itemize}
\end{frame}


\begin{frame}[fragile]\frametitle{\insertsection}
	\framesubtitle{\insertsubsection}
	\structure{Beispiel einer Containerklasse:}
	\begin{lstlisting}[xleftmargin=3ex]
package datamodel; 

public class Auto {
   private int auto_id; 
	
   private String farbe; 
	
   private int anzahlRaeder; 
	
   public int getId() { return this.auto_id; }

   public void setId(int id) { this.id = id;  }

...
   
}
\end{lstlisting}
\end{frame}

\begin{frame}[fragile]\frametitle{\insertsection}
	\framesubtitle{\insertsubsection}
	\structure{Beispiel Zugriffsinterface}
	\begin{lstlisting}[xleftmargin=3ex]
package repository; 
	
public interface IRepository {
  public Auto createAuto(); 
		
  public Auto[] lookupAutoByFarbe(String farbe); 
  		
  public Auto lookupAutoById(int id); 
		
  public void saveAuto(Auto a); 	
}
	\end{lstlisting}
\end{frame}

\begin{frame}[fragile]\frametitle{\insertsection}
	\framesubtitle{\insertsubsection}
	\structure{Beispiel Implementierungsklasse (Pseudocode)}
	\begin{lstlisting}[xleftmargin=3ex]
package repository; 
public class PostgresqlRepository implements IRepository {
   private Connection _con; 
  ...
   public Auto lookupAutoById(int id) {
      PreparedStatement pst = 
         new PreparedStatement("SELECT * from Auto WHERE id = ?"); 
      pst.setInt(1,id); 
      ResultSet rs = pst.execute(); 
	  Auto a = new Auto; 
      while (rs.next()) {
	        a.setId(rs.getInt(0));
	        a.setFarbe(rs.getFarbe(1));
	        ...
      }
   return a;
   }
}
\end{lstlisting}
\end{frame}

\begin{frame}[fragile]\frametitle{\insertsection}
	\framesubtitle{\insertsubsection}
	\structure{Factory-Klasse (Pseudocode)}
	\begin{lstlisting}[xleftmargin=3ex]
package repository; 
public class RepoFactory {
   ...
   public static IRepository createRepository() {
      if (config.getDatabase() == 'Postgresql') {
         ...
         return new PostgresqlRepository(); 
      } else if (config.getDatabase() == 'MySQL') {
          ...
          return new MysqlRepository(); 
      }
   }
}
	\end{lstlisting}
	Es wird davon ausgegangen, dass im Objekt \texttt{config} eine Konfigurationsdatei eingelesen wurde, in der die Datenbank konfiguriert wurde.
\end{frame}

\begin{frame}[fragile]\frametitle{\insertsection}
	\framesubtitle{\insertsubsection}
	\structure{Verwendung der Klassen:}
	\begin{lstlisting}[xleftmargin=3ex]
public static void main(String args[]) {
   
   IRepository repo = RepoFactory.createRepository(); 
   Auto auto = repo.lookupAutoById(4711);

   Auto nocheinauto = new Auto(); 
   nocheinauto.setFarbe("schwarz");
   repo.saveAuto(nocheinauto);   
}
	\end{lstlisting}
\end{frame}

%
%\section{Objekt-relationales Mapping}
%\subsection{Begriffserklärung}
%
%\begin{frame}[t]\frametitle{Objekt-relationales Mapping}
%    \framesubtitle{Begriffserklärung}
%    \structure{In der objektorientierten Welt sind Objekte meist start strukturiert.}
%    \begin{itemize}
%    	\item Starke Schachtelung von Objekten
%    	\item Polymorphie und abstrakte Klassen
%    	\item Schnittstellenkonzepte
%    	\item Vererbungshierarchien 
%    \end{itemize}
%Diese Strukturen müssen mittels \textit{object flattening} auf flache Tabellen abgebildet werden und wieder zurückgewandelt werden können.
%	\begin{definition}[Object-Relational Impedance Mismatch]
%	Unter dem Begriff wird die grundlegende Unverträglichkeit von objektorientierten Programmiersprachen mit den Konzepten relationaler Datenbanken verstanden. Es 
%	besteht ein \textit{Paradigmakonflikt}.	
%	\end{definition}
%\end{frame}
%
%\againframe<2>{isa}
%
%\begin{frame}[t]\frametitle{Objekt-relationales Mapping}
%	\framesubtitle{Ansätze für die Abbildung von is-a -- Beziehungen}
%	\structure{Es wurden verschiedene Möglichkeiten diskutiert:}
%	\begin{itemize}
%		\item One Table per Class Hierarchy 
%		\begin{itemize}
%			\item \texttt{null}-Values 
%			\item Normalisierung? 
%		\end{itemize}
%		\item One Table per Concrete Class 
%		\begin{itemize}
%			\item Informationen der Vererbung gehen verloren: Abstrakte Klasse existiert nicht mehr im Datenmodell, Unterscheidung, welches Attribut zur abstrakten Klasse und zur konkreten Klasse gehört ist umständlich
%		\end{itemize}
%		\item One Table per Class
%		\begin{itemize}
%			\item Alle Informationen vorhanden, aber erhöhter Join-Aufwand 
%		\end{itemize}
%	\end{itemize}
%   \alert{Da es sich hier um ein grundsätzliches Abbildungsproblem handelt, gibt es keine einheitliche Lösung -- bestenfalls eine Vereinfachung für den Entwickler.}
%\end{frame}
%
%\begin{frame}[t]\frametitle{Objekt-relationales Mapping}
%	\framesubtitle{Verschiedene Konzepte}
%	Da ein Anwendungsentwickler auf Objekten arbeitet, musst die Speicherung der Daten (respektive die Zerlegung der Objekte in Relationen) manuell durchgeführt werden.
%	Es gibt verschiedene Ansätze, dies zu vereinfachen
%	\begin{itemize}
%		\item Der Einsatz objektorientierter Datenbanken 
%		\begin{itemize}
%			\item Objekte konnten direkt in der Datenbank abgelegt werden
%			\item Nicht standardisiert, Auslesen der Objekte nur mit der Anwendung möglich
%			\item Kommen in speziellen Anwendungsfällen vor
%		\end{itemize}
%		\item Erweiterung objektorientierter Programmiersprachen um relationale Konzepte
%		\begin{itemize}
%			\item SQL wird direkt im Code eingebettet
%			\item Lösung widerspricht in großen Teilen den objektorientierten Ansätzen
%			\item Wird in der heutigen Entwicklung fast nicht mehr eingesetzt
%		\end{itemize}
%	\end{itemize}
%\end{frame}
%
%
%\subsection{O/R-Mapper}
%
%\begin{frame}[t]\frametitle{Objekt-relationales Mapping}
%	\framesubtitle{O/R-Mapper Allgemein}
%	\structure{O/R-Mapper bilden eine Zwischenschicht zwischen Anwendung und Datenbank}
%	\begin{itemize}
%		\item Sie erledigen das Mapping von Objekten auf Relationen 
%		\item Transparent für den Anwendungsentwickler
%		\item Weite Verbreitung 
%		\item Viele unterschiedliche Implementierungen und Standards 
%		\begin{itemize}
%			\item Microsoft .NET Entity Framework 
%			\begin{itemize}
%				\item Alle .NET-Programmiersprachen (C\#, Visual Basic, Visual C++)
%				\item Gute Unterstützung verschiedener Datenbanken
%				\item Unterstützt Code-First und Model-First Ansätze
%			\end{itemize}
%			\item Java Persistence API (JPA)
%			\begin{itemize}
%				\item Standardisierte O/R-Mapper-Spezifikation für Java
%			\end{itemize}
%		\end{itemize}
%	\end{itemize}
%\end{frame}
%
%
%\begin{frame}[t]\frametitle{Objekt-relationales Mapping}
%	\framesubtitle{Java Persistence API (JPA)}
%	\structure{Die JPA wurde erstmals in 2006 veröffentlicht}
%	\begin{itemize}
%		\item Standardisierte Programmierschnittstelle (API)
%		\begin{itemize}
%			\item Bietet Methoden zum Speichern, Modifizieren und Laden von Objekten aus relationalen Datenbanken
%			\item Unterstützung von Polymorphie / Vererbung 
%			\item Verwendet Java Annotations (z.B. \texttt{@Entity, @OneToMany})
%		\end{itemize}
%		\item Java Persistence Query Language 
%		\begin{itemize}
%			\item Eigene, an SQL angelehnte Sprache
%		\end{itemize}
%		\item XML-Datei zur Konfiguration des relationalen Mappings
%		\begin{itemize}
%			\item Mittels einer \texttt{persistence.xml}-Datei kann das Mapping genau definiert werden
%		\end{itemize}
%		\item Viele Implementierungen der JPA, z.B. \textit{Eclipselink} oder \textit{Hibernate}
%	\end{itemize}
%\end{frame}
%
%
%\begin{frame}[t]\frametitle{Objekt-relationales Mapping}
%	\framesubtitle{EclipseLink}
%	\structure{JPA-Referenzimplementierung \textit{EclipseLink}}
%	\begin{itemize}
%		\item Kann unter \url{https://www.eclipse.org/eclipselink/} heruntergeladen werden
%		\item Umfasst noch mehr als nur eine JPA-Implementierung -- in der Vorlesung wird jedoch nur auf den JPA-Teil eingegangen
%		\item Konfiguration über \texttt{persistence.xml}
%		\item Verwendung von Annotationen im Code
%		\item Muss im \texttt{CLASSPATH} vorhanden sein
%	\end{itemize}
%\end{frame}
%
%
%
%\begin{frame}[fragile]\frametitle{Objekt-relationales Mapping}
%    \framesubtitle{Beispiel persistence.xml}
%    \lstset{language=xml}
%	\lstset{escapeinside={(*@}{@*)}}
%
%	\begin{lstlisting}[xleftmargin=3ex]
%<?xml version="1.0" encoding="UTF-8" ?>
%<persistence xmlns:xsi=...> 
% <persistence-unit name="filme" transaction-type="RESOURCE_LOCAL">   
% <class>de.test.Film</class>   
% <class>de.test.Genre</class>   
% <properties>     
%  <property name="javax.persistence.jdbc.driver" value="com.mysql.jdbc.Driver"/>     
%  <property name="javax.persistence.jdbc.url" value="jdbc:mysql://localhost/wwidb1"/>
%  <property name="javax.persistence.jdbc.user" value="db"/>
%  <property name="javax.persistence.jdbc.password" value="db"/>
%  <property name="eclipselink.ddl-generation" value="create-tables"/>
%  <property name="eclipselink.ddl-generation.output-mode" value="database"/>
% </properties>
%</persistence-unit>
%</persistence> 
%	\end{lstlisting}
%\end{frame}
%
%\begin{frame}[fragile]\frametitle{Objekt-relationales Mapping}
%    \framesubtitle{Beispiel Java-Code}
%    \lstset{language=Java}
%	\lstset{escapeinside={(*@}{@*)}}
%
%	\begin{lstlisting}[xleftmargin=3ex, basicstyle=\ttfamily\tiny]
%package de.test;
%import javax.persistence.CascadeType;
%import javax.persistence.Entity;
%import javax.persistence.GeneratedValue;
%import javax.persistence.GenerationType;
%import javax.persistence.Id;
%import javax.persistence.OneToOne;
%
%@Entity
%public class Film {
%
%@Id
%@GeneratedValue(strategy=GenerationType.IDENTITY)
%private long id ; 
%
%private String name ; 
%
%private int jahr;
%
%@OneToOne(cascade=CascadeType.PERSIST)
%private Genre genre ; 
%
%//Getter und Setter fuer private Attribute
%...
%}
%\end{lstlisting}
%\end{frame}

%\section*{Zusammenfassung}
%
%\begin{frame}{Zusammenfassung}
%	\begin{itemize}
%		\item Grundlagen über Systemarchitekturen
%		\begin{itemize}
%					\item 2-Tier
%					\item 3-Tier
%		\end{itemize}		
%		\item JDBC
%		\begin{itemize}
%			\item Einführung
%			\item API-Struktur
%			\item Connection Pooling
%		\end{itemize}
%		\item Object-relationaler Impedance Mismatch und O/R - Mapper
%	\end{itemize}
%
%\end{frame}
%