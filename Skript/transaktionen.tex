%!TEX root = Slides.tex
\part{Transaktionen}

\section{Grundlagen und Begriffsdefinitionen}

\subsection{Datenbankobjekte}
\begin{frame}{\insertsection}
	\framesubtitle{\insertsubsection}
	\structure{\textit{Datenbankobjekte}}	
	\begin{itemize}
		\item Datenbankobjekte haben Bezeichner und k\"onnen Attribute, Datens\"atze, Bl\"ocke, Relationen usw. sein
		\item Die Granularität spielt in der Betrachtung von Transaktionen keine Rolle
	\end{itemize}
	\structure{Operationen auf Datenbankobjekten}
	\begin{itemize}
		\item \texttt{read\_item(X)} liest ein Objekt der Datenbank mit dem Namen \texttt{X} in eine Programmvariable \texttt{X}
		\item \texttt{write\_item(X)} schreibt den Wert der Programmvariablen \texttt{X} in das Objekt \texttt{X} der Datenbank
	\end{itemize}
\end{frame}

\begin{frame}{\insertsection}
	\framesubtitle{\insertsubsection}
	\structure{Vorgehen bei \texttt{read\_item(X)}:}	
	\begin{enumerate}
		\item Suche die Adresse des Festplattenblocks, der das Objekt \texttt{X} enthält
		\item Kopiere diesen Plattenblock in den Puffer im Arbeitsspeicher (sofern nicht bereits dort enthalten)
		\item Kopiere Objekt \texttt{X} vom Puffer in die Programmvariable namens \texttt{X}
	\end{enumerate}
\end{frame}

\begin{frame}{\insertsection}
\framesubtitle{\insertsubsection}
\structure{Vorgehen bei \texttt{write\_item(X)}:}	
\begin{enumerate}
	\item Suche die Adresse des Festplattenblocks, der das Objekt \texttt{X} enthält
	\item Kopiere diesen Plattenblock in den Puffer im Arbeitsspeicher (sofern nicht bereits dort enthalten)
	\item Kopiere Objekt \texttt{X} von der Programmvariable \texttt{X} an die richtige Stelle im Puffer
	\item Schreibe den aktualisierten Block aus dem Puffer (sofort oder später) auf Platte zurück 
\end{enumerate}
\end{frame}

\subsection{Transaktionen}
\begin{frame}{\insertsection}
\framesubtitle{\insertsubsection}
\structure{\textbf{Transaktion}}
\begin{itemize}
	\item beschreibt logische Verarbeitungseinheit, die aus Sequenz von Einzeloperationen besteht
	\item wird entweder vollständig oder gar nicht ausgeführt 
	\item \"uberf\"uhrt Datenbank von konsistentem Zustand in neuen, konsistenten Zustand 
	\begin{itemize}
		\item konsistenter Zustand erf\"ullt die Integritätsbedingungen des Schemas
	\end{itemize}
\end{itemize}
\end{frame}

\begin{frame}
	\frametitle{\insertsection}
	\framesubtitle{\insertsubsection}
	\structure{Geforderte Eigenschaften von Transaktionen: ACID}
	\begin{itemize}
		\item \textbf{A}tomicity: Unteilbarkeit der Operationen
		\item \textbf{C}onsistency: Jede Transaktion hinterlässt die Datenbank in konsistentem Zustand
		\item \textbf{I}solation: Transaktionen werden ohne gegenseitige Beeinflussung ausgeführt
		\item \textbf{D}urability: Änderungen abgeschlossener Transaktionen werden dauerhaft in der DB persistiert
	\end{itemize}
	\vspace{1em}	
ACID ist für unternehmenskritische OLTP-Anwendungen essentiell und muss vom DBMS sichergestellt werden.	
\end{frame}

\begin{frame}{\insertsection}
\framesubtitle{\insertsubsection}
\structure{Transaktion als Einheit mehrerer DB-Operationen}
\begin{itemize}
	\item Transaktionen fassen DB-Operationen zu einer Einheit zusammen.
	\item In SQL werden Transaktionen explizit oder implizit mit den Befehlen
	\begin{itemize}
		\item \texttt{BEGIN TRANSACTION} gestartet (Start)
		\item \texttt{COMMIT} abgeschlossen (Ende)
		\item \texttt{ROLLBACK} r\"uckg\"angig gemacht (Ende)
	\end{itemize}
	\pause
	\item Alle Operationen zwischen Start und Ende werden als eine Einheit aufgefasst. 
	\item Sie werden dann entweder ganz (\texttt{COMMIT}) oder gar nicht (\texttt{ROLLBACK}) ausgeführt.
	\pause
	\item Transaktionen werden auch implizit durch die DB gestartet und beendet.
	\begin{itemize}
		\item DB entscheidet dann, ob Transaktion durchgef\"uhrt oder r\"uckg\"angig gemacht wird.
	\end{itemize}
\end{itemize}
\abs\pause
\textbf{Folge:} Bei Scheitern einer Transaktion muss urspr\"unglicher Datenbankzustand wiederhergestellt werden
$\Rightarrow$ Recovery-Konzepte nötig (siehe PDF-Seite \pageref{sec:recovery}).
\end{frame}

\begin{frame}{\insertsection}
\framesubtitle{\insertsubsection}
\structure{Bei Scheitern einer Transaktion: Wiederherstellung des vorigen \textit{Datenbankzustands}}
\abs
\pause
Gründe für Fehlschlagen einer Transaktion:
\begin{itemize}
\item Rechnerfehler oder Systemabstürze -- z.~B.~Ausfall des Arbeitsspeichers, Netzwerkfehler\\[4pt]
\item Transaktionsfehler -- z.~B.~durch eine Division durch Null oder Integer-Überlauf\\[4pt]
\item Lokale Fehler -- z.~B.~können notwendige Daten für die Durchführung einer Transaktion nicht gefunden werden\\[4pt]
\item Abbruch einer Transaktion durch Nebenläufigkeitskontrolle (dazu später mehr)\\[4pt]
\item Plattenfehler -- z.~B.~durch einen Crash des Schreib/Lesekopfes der Platte \\[4pt]
\item Externe Probleme -- z.~B.~Ausfall der Stromversorgung, Diebstahl, Katastrophen \\[4pt]
\end{itemize}
\end{frame}

\subsection{Parallelität}
\begin{frame}{\insertsection}
\framesubtitle{\insertsubsection}
\structure{Nahezu alle DBMS arbeiten im Mehrbenutzerbetrieb}
\begin{itemize}
	\item Das kann zu Konflikten führen!
	\item Lesender Zugriff von Benutzer $A$, während Benutzer $B$ einen Datensatz neu schreibt 
	\item Gleichzeitiger schreibender Zugriff der Benutzer $A$ und $B$
\end{itemize}
\pause\abs
\structure{Paralleler Zugriff auf Daten wird durch moderne CPU-Architekturen ermöglicht.}
\end{frame}

\subsection{Problemstellung bei paralleler Ausführung von Transaktionen}

\subsection{Parallelität}
\begin{frame}{\insertsection}
\framesubtitle{\insertsubsection}
Mehrere Prozesse greifen zeitgleich auf die Datenbank zu: 
	\begin{itemize}
		\item Interleaved Execution (siehe Abbildung zwischen $t_1$ und $t_2$)
		\item Parallel Execution (siehe Abbildung zwischen $t_3$ und $t_4$)
	\end{itemize}
\begin{figure}
	\includegraphics[width=250pt]{img/parallel-processes.png}
	\caption{Interleaved vs. parallel execution (nach \cite[S. 745]{EN10})}
\end{figure}
\end{frame}

\begin{frame}{\insertsection}
\framesubtitle{\insertsubsection}
\structure{Seien $T_1$ und $ T_2$ zwei Transaktionen, die unabhängig voneinander ausgeführt werden k\"onnen}		
\begin{center}
	\begin{tabular}{|p{3.5cm}|}
		\multicolumn{1}{c}{$T_1$} \\\hline
		\texttt{read\_item(X)} \\
		\texttt{X:=X-N} \\
		\texttt{write\_item(X)}\\
		\texttt{read\_item(Y)}  \\
		\texttt{Y:=Y+N} \\
		\texttt{write\_item(Y)}\\\hline
	\end{tabular}
	\hspace{0.5cm}
	\begin{tabular}{|p{3.5cm}|}
		\multicolumn{1}{c}{$T_2$} \\\hline
		\texttt{read\_item(X)}\\
		\texttt{X:=X+M}\\
		\texttt{write\_item(X)}\\
		\\
		\\
		\\\hline
	\end{tabular}
\end{center}
\begin{itemize}
	\item $X$ repr\"asentiert die freien Pl\"atze eines Fluges 
	\item $Y$ stellt die Platzbelegungen dar
	\item $T_1$ vermindert die Anzahl freier Pl\"atze nach Reservierung
	\item $T_2$ erh\"oht die Anzahl freier Pl\"atze nach Stornierung
\end{itemize}
\pause
\textbf{Welches Problem kann hier unmittelbar erkannt werden?}
\end{frame}

\begin{frame}{\insertsection}
\framesubtitle{\insertsubsection}
\structure{Das Lost-Update-Problem. Parameter: $N=5, M=4$}
\begin{center}
\begin{tabular}{c|p{3.5cm}|c|p{3.5cm}|c}
	$X_{T_1}$ & \multicolumn{1}{c|}{$T_1$} & $X$ & \multicolumn{1}{c|}{$T_2$} & $X_{T_2}$ \\\hline
	80 & \texttt{read\_item(X)} & 80 & &\\
	75 & \texttt{X:=X-N} & & & \\
	& & & \texttt{read\_item(X)} & 80 \\
	& & & \texttt{X:=X+M} & 84\\
	& \texttt{write\_item(X)} & 75 & & \\
	& \texttt{read\_item(Y)} &  & & \\
	& & \cellcolor{Red}84 & \texttt{write\_item(X)} & \\
	& \texttt{Y:=Y+N} &  & & \\
	& \texttt{write\_item(Y)} &&&\\
\end{tabular}
\end{center}
\textbf{Wie viele Personen sitzen wirklich im Flugzeug?}
\end{frame}

\begin{frame}{\insertsection}
	\framesubtitle{\insertsubsection}
	\begin{columns}
	\begin{column}{.48\textwidth}
	\begin{tabular}{p{3.0cm}|p{3.0cm}}
			\multicolumn{1}{c|}{$T_1$} & \multicolumn{1}{c}{$T_2$}\\\hline
			\texttt{read\_item(X)} & \\
			\texttt{X:=X-N} & \\ 
			\texttt{write\_item(X)} & \\
			& \texttt{read\_item(X)} \\
			& \texttt{X:=X+M}\\
			& \texttt{write\_item(X)}\\
			\texttt{read\_item(Y)} & \\
			\texttt{Y:=Y+N}& \\
			\cellcolor{Yellow}\texttt{write\_item(Y)} & \\\hline
			\end{tabular}
		\end{column}
		\begin{column}{.48\textwidth}
		\structure{Dirty Read Problem}
		\begin{itemize}
			\item $T_1$ liest einen Wert und schreibt ein Update 
			\item $T_2$ liest den \textit{neuen} Wert und schreibt ebenfalls ein Update 
			\item $T_1$ schlägt fehl (z.B. weil Y nicht geschrieben werden kann). Damit wird $T_1$ abgebrochen und der ursprüngliche Wert von $X$ muss wiederhergestellt werden. \alert{Aber welcher?}
		\end{itemize}
		\end{column}
	\end{columns}
	\end{frame}

\begin{frame}{\insertsection}
	\framesubtitle{\insertsubsection}
	\begin{columns}
	\begin{column}{.48\textwidth}
	\begin{tabular}{p{3.0cm}|p{3.0cm}}
			\multicolumn{1}{c|}{$T_1$} & \multicolumn{1}{c}{$T_3$}\\\hline
			& \texttt{sum:=0}\\
			& \texttt{read\_item(A)}\\
			\texttt{read\_item(X)} & \texttt{sum:=sum+A}\\
			\texttt{X:=X-N} & \multicolumn{1}{c}{$\vdots$}  \\
			\texttt{write\_item(X)} & \\ 
			& \texttt{read\_item(X)} \\
			& \texttt{sum:=sum+X}\\
			& \texttt{read\_item(Y)}\\
			& \texttt{sum:=sum+Y} \\
			\texttt{read\_item(Y)}& \\
			\texttt{Y:=Y+N} & \\
			\texttt{write\_item(Y)} & \\\hline
			\end{tabular}
		\end{column}
		\begin{column}{.48\textwidth}
		\structure{Non-Repeatable Read/falsche Summenbildung}
		\begin{itemize}
			\item $T_3$ liest $X$, nachdem $N$ abgezogen wurde 
			\item $T_3$ liest $Y$, bevor $N$ addiert wurde 
		\end{itemize}
		\alert{Das Ergebnis weicht um $N$ vom korrekten Wert ab.}
		\end{column}
	\end{columns}
	\end{frame}

\begin{frame}{\insertsection}
	\framesubtitle{\insertsubsection}
	\begin{columns}
		\begin{column}{.48\textwidth}
			\begin{tabular}{p{3.0cm}|p{3.0cm}}
				\multicolumn{1}{c|}{$T_1$} & \multicolumn{1}{c}{$T_3$}\\\hline
				SELECT count(*) INTO X FROM Y; & \\
				& INSERT INTO Y VALUES (...);\\
				& commit;\\
				UPDATE Y SET Y.a = Y.a + (1/X) &   \\\hline
			\end{tabular}
		\end{column}
		\begin{column}{.48\textwidth}
			\structure{Phantom Read}
			\begin{itemize}
				\item $T_3$ fügt einen Phantomdatensatz ein, der von $T_1$ nicht berücksichtigt wird. 
			\end{itemize}
		\end{column}
	\end{columns}
\end{frame}

\begin{frame}{\insertsection}
\framesubtitle{\insertsubsection}
Moderne Datenbanken haben Konzepte und Mechanismen, wodurch
\begin{itemize}
	\item die genannten Probleme gel\"ost werden k\"onnen.
	\item gescheiterte Transaktionsketten erfolgreich konsistent r\"uckg\"angig gemacht werden k\"onnen.
\end{itemize}
\abs
$\Rightarrow$ Recovery und Nebenläufigkeitskontrolle.
\end{frame}

\section{Recovery und Nebenläufigkeitskontrolle}\label{sec:recovery}
\subsection{Recovery Manager}

\begin{frame}{\insertsection}
\framesubtitle{\insertsubsection}
\structure{Recovery Manager überwacht die Zust\"ande der Transaktionen}
\abs
Folgende Informationen sind dem Recovery Manager bekannt:
\begin{itemize}
	\item Start einer Transaktion (\texttt{BEGIN TRANSACTION})
	\item Welche \texttt{READ}- und \texttt{WRITE}-Operationen von einer Transaktion durchgeführt werden
	\item Wann die Kette von \texttt{READS} und \texttt{WRITES} beendet ist (\texttt{END\_TRANSACTION})
\end{itemize}
\pause
\abs
Aufgabe des Recovery Manager 
\begin{itemize}
	\item Nach erfolgreicher Prüfung wird dies durch \texttt{COMMIT} angezeigt. 
	Änderungen werden auf Platte geschrieben -- ggf.~nochmaliges Pr\"ufen (OS-Prozess) vor \texttt{COMMIT}
	\item Fehlerhaftes Ende einer Transaktion wird durch \texttt{ROLLBACK} (oder \texttt{ABORT}) angezeigt.
	Änderungen werden nicht auf Platte geschrieben und müssen rückgängig gemacht werden.
\end{itemize}
\end{frame}

\begin{frame}{\insertsection}
	\framesubtitle{\insertsubsection}
\begin{figure}
\begin{tikzpicture}
\ttfamily
\node[] (init){}; 
\node[draw, rectangle, rounded corners=1mm, minimum size=1.5cm] (start) [right= 2cm of init] {\tiny Active}; 
\draw[->] (init) -- (start) node[midway, above, align=center] {\tiny BEGIN TRANS.}; 
\path[draw] (start) edge[loop above] () node[above=9mm of start] {\tiny READ/WRITE}; 
\node[draw, rectangle, rounded corners=1mm,  minimum size=1.5cm, align=center] (pc) [right=2cm of start] {\tiny Partially\\ \tiny Committed};
\draw[->] (start) -- (pc) node[midway, above] {\tiny END TRANS.}; 
\node[draw, rectangle, rounded corners=1mm,  minimum size=1.5cm] (c) [right=2cm of pc] {\tiny Committed};
\node[draw, rectangle, rounded corners=1mm,  minimum size=1.5cm] (f) [below=of pc] {\tiny Failed};
\draw[->] (pc) -- (c) node[midway, above] {\tiny COMMIT}; 
\draw[->] (pc) -- (f) node[midway, right] {\tiny ABORT};
\draw[->] (start) |- (f) node[midway, left] {\tiny ABORT}; 
\node[draw, rectangle, rounded corners=1mm,  minimum size=1.5cm] (t) [below=of c] {\tiny Terminated};
\draw[->] (c) -- (t); 
\draw[->] (f.east) -- (t.west); 
\end{tikzpicture}
\caption{Zustandsübergangsdiagramm einer Transaktion}
\end{figure}
\end{frame}

\begin{frame}{\insertsection}
	\framesubtitle{\insertsubsection}
\structure{Um Transaktionen rückgängig machen zu können, müssen die Operationen einer Transaktion 
	protokolliert werden: Database Log}
\begin{itemize}
\item Database-Log ist sequentielle Append-Only-Datei
\pause
\item Alle Operationen einer Transaktion $T$ und einem Objekt $X$ werden im Log gespeichert: 
\begin{itemize}
\item Start einer Transaktion: \texttt{[start\_transaction, T]}
\item Write-Operationen: \texttt{[write\_item, T, X, alt, neu]}
\item Read-Operationen: \texttt{[read\_item, T, X]} 
\item Abbruch einer Transaktion: \texttt{[abort, T]}
\item Commit einer Transaktion: \texttt{[commit, T]} (wird als \textit{Commit Point} bezeichnet)
\end{itemize}
\pause
\item Bei Auftreten eines Fehlers kann m.~H.~des Logs die Transaktion rückgängig gemacht werden
\pause
\item Transaktion hinter ihrem Commit Point ist permanent in der Datenbank gespeichert
\end{itemize} 
\end{frame}

\subsection{Ausführungspläne}

\begin{frame}{\insertsection}
\framesubtitle{\insertsubsection}
\begin{definition}[Ausführungsplan (Schedule)]
Ein Ausführungsplan $S(T_1,T_2,\dots ,T_n)$ von Transaktionen $T_1,T_2,\dots ,T_n$ ist eine Sequenz der 
Operationen aller Transaktionen $T_i$ mit der Eigenschaft, dass die Operationen in $S$ in der gleichen 
Reihenfolge erscheinen müssen wie in $T_i$. 
\end{definition}
\begin{itemize}
\item Die Operationen der in $S$ beteiligten Transaktionen können sich vermischen, solange die o.g. Reihenfolge eingehalten wird.
\item Operationen \texttt{read}, \texttt{write}, \texttt{abort}, \texttt{commit} werden mit \texttt{r}, \texttt{w}, 
\texttt{c}, \texttt{a} abgekürzt
\end{itemize}
\abs
\textbf{Vom Ausführungsplan hängt sowohl die Nebenläufigkeit als auch das Recovery ab.}
\end{frame}

\begin{frame}{\insertsection}
	\framesubtitle{\insertsubsection}
	\begin{figure}
		\includegraphics[scale=0.5]{img/scheduler.pdf}
	\end{figure}
\end{frame}

\begin{frame}{\insertsection}
	\framesubtitle{\insertsubsection}
\begin{center}
\begin{tabular}{c|p{3.5cm}|c|p{3.5cm}|c}
	$X_{T_1}$ & \multicolumn{1}{c|}{$T_1$} & $X$ & \multicolumn{1}{c|}{$T_2$} & $X_{T_2}$ \\\hline
	80 & \texttt{read\_item(X)} & 80 & &\\
	75 & \texttt{X:=X-N} & & & \\
	& & & \texttt{read\_item(X)} & 80 \\
	& & & \texttt{X:=X+M} & 84\\
	& \texttt{write\_item(X)} & 75 & & \\
	& \texttt{read\_item(Y)} & & & \\
	& & \cellcolor{Red}84 & \texttt{write\_item(X)} & \\
	& \texttt{Y:=Y+N} &  & & \\
	& \texttt{write\_item(Y)} &&&\\
	\end{tabular}
\end{center}
Ausführungsplan: 
$$S: r_1(X) - r_2(X) - w_1(X) - r_1(Y) - w_2(X) - w_1(Y)$$
\end{frame}

\begin{frame}{\insertsection}
\framesubtitle{\insertsubsection}
\begin{definition}[Konflikt im Ausf\"uhrungsplan]
	\structure{Zwei Operationen eines Ausführungsplanes stehen in Konflikt, wenn folgende Bedingungen erf\"ullt sind:}
	\begin{enumerate}
		\item Die Operationen gehören zu unterschiedlichen Transaktionen.
		\item Die Operationen greifen auf das gleiche Objekt zu.
		\item Mindestens eine der Operationen ist \texttt{write}.
	\end{enumerate}
\end{definition}
\pause
\abs
Beispiel: 
$$\text{Ausführungsplan }S: r_1(X) - r_2(X) - w_1(X) - r_1(Y) - w_2(X) - w_1(Y)$$
\textbf{Konflikte: $(r_1(X), w_2(X))$ und $(w_1(X), w_2(X))$}.
\end{frame}

\begin{frame}{\insertsection}
\framesubtitle{\insertsubsection}
\structure{\textbf{Beachte:} Der folgende Ausf\"uhrungsplan hat in Konflikt stehende Operationen.}
{\small
	\begin{center}
		\begin{tabular}{p{3.0cm}|p{3.0cm}}
			\multicolumn{1}{c|}{$T_1$} & \multicolumn{1}{c}{$T_2$}\\\hline
			\texttt{read\_item(X)} & \\
			\texttt{X:=X-N} & \\ 
			\texttt{write\_item(X)} & \\
			\texttt{read\_item(Y)} & \\
			\texttt{Y:=Y+N} &\\
			\texttt{write\_item(Y)} & \\
			&  \texttt{read\_item(X)}\\
			&  \texttt{X:=X+M}\\
			&\texttt{write\_item(X)} \\\hline
		\end{tabular}
	\end{center}
}
\pause
\textbf{Aber:} 
\begin{itemize}
	\item Die Transaktionen werden in nicht \"uberschneidenden Zeitr\"aumen ausgef\"uhrt.
	\item Wegen der ACID-Annahme: Datenbank daher am Ende in konsistentem Zustand.
	\item Ein solcher Ausf\"uhrungsplan hei\ss t \textit{seriell}.
\end{itemize}
\end{frame}

\begin{frame}{\insertsection}
\framesubtitle{\insertsubsection}
\begin{definition}[Serieller Ausführungsplan]
	Ein Ausführungsplan $S(T_1,T_2,\dots ,T_n)$ hei\ss t seriell, wenn f\"ur alle Transaktionen $T_i$ und $T_j$ mit $i\ne j$ 
	eine der beiden Bedingungen gilt:
	\begin{itemize}
		\item Alle Operationen von $T_i$ sind in $S$ vor den Operationen von $T_j$ angeordnet.
		\item Alle Operationen von $T_i$ sind in $S$ nach den Operationen von $T_j$ angeordnet.
	\end{itemize}
\end{definition}
\nl
\pause
\textbf{Folge}:
\\[4pt]
In einem seriellen Ausführungsplan $S(T_1,T_2,\dots ,T_n)$ werden die Transaktionen $T_1,T_2,\dots ,T_n$ in einer bestimmten, 
permutierten Reihenfolge hintereinander ausgef\"uhrt: 
\begin{equation*}
S= T_{\sigma(1)}-T_{\sigma(2)}-\ldots -T_{\sigma(n)}
\end{equation*}
\end{frame}

\begin{frame}{\insertsection}
\framesubtitle{Serieller Ausführungsplan}
\begin{itemize}
\item In seriellem Ausf\"uhrungsplan werden daher die Transaktionen in nicht \"uberschneidenden Zeitr\"aumen ausgef\"uhrt.
\item Wegen ACID-Annahme: Datenbank nach Ausf\"uhrung von $S$ in konsistentem Zustand.	
\end{itemize}
\ \\[4pt]
\pause
\textbf{Aber:} Nebenl\"aufigkeit ist stark eingeschr\"ankt. Im allgemeinen nicht akzeptabel!
\end{frame}

\begin{frame}{\insertsection}
\framesubtitle{\insertsubsection}
\textbf{Ziel:}	Finde Ausf\"uhrungspl\"ane, die 
\begin{itemize}
	\item zu gleichem Ergebnis wie ein serieller Ausf\"uhrungsplan f\"uhren
	\item und dennoch ggf.~Nebenl\"aufigkeit erm\"oglichen
\end{itemize}
\pause
\abs
Wir werden zeigen: 
\\[4pt]
Serialisierbare Ausf\"uhrungspl\"ane f\"uhren zu gleichem Ergebnis wie ein serieller Ausf\"uhrungsplan.
\pause
\\[10pt]
Serialisierbarkeit und Nebenl\"aufigkeitskontrolle
\begin{itemize}
	\item Serialisierbarkeit gew\"ahrleistet hohen Grad an Isolation der Operationen auf DB-Objekten.
	\item Viele Konzepte der Nebenl\"aufigkeitskontrolle (siehe sp\"ater) f\"uhren zu Interferenzfreiheit und Isolation
	von nebenl\"aufig ausgef\"uhrten Transaktionen. 
	\item Die meisten dieser Konzepte stellen Serialisierbarkeit der Ausf\"uhrungspl\"ane sicher.
\end{itemize}
\end{frame}

\begin{frame}{\insertsection}
\framesubtitle{\insertsubsection}
\begin{definition}[Serialisierbarer Ausführungsplan]
Ein Ausführungsplan $S(T_1,T_2,\dots ,T_n)$ hei\ss t serialisierbar, wenn die Reihenfolge von 
zwei beliebigen in Konflikt stehenden Operationen $O_1$ und $O_2$ in $S$ mit der Reihenfolge  
dieser Operationen in einem seriellen Ausführungsplan $S_{ser}(T_1,T_2,\dots ,T_n)$ \"ubereinstimmt.
\abs
Bezeichnung: $S$ ist konflikt-\"aquivalent zu $S_{ser}$.
\nl
Notation: $S\equiv S_{ser}$.
\end{definition}
\pause
\abs
\begin{theorem}
Sei $S(T_1,T_2,\dots ,T_n)$ ein serialisierbarer Ausführungsplan mit $S\equiv S_{ser}$ f\"ur einen 
seriellen Ausführungsplan $S_{ser}(T_1,T_2,\dots ,T_n)$.
\abs
Dann f\"uhrt die Ausf\"uhrung von $S$ zum gleichen Ergebnis wie die Ausf\"uhrung von $S_{ser}$.
\end{theorem}
\end{frame}

\begin{frame}{\insertsection}
\framesubtitle{\insertsubsection}
\textbf{Ziel:}	Finde Kriterium f\"ur Serialisierbarkeit. 
\pause
\ \\[4pt]
\begin{block}{\textbf{Serialisierbarkeitstest}}
	Sei $S(T_1,T_2,\dots ,T_n)$ ein Ausführungsplan.
	\begin{enumerate}
		\item F\"ur jede Transaktion $T_i$, $i\in\{1,\ldots,n\}$ lege einen mit $T_i$ bezeichneten Knoten an.
		\pause
		\item F\"ur jedes Paar von im Ausf\"uhrungsplan $S$ in Konflikt stehenden Operationen $O_{j}(X)$ von $T_j$ und 
		$O_{k}(X)$ von $T_k$, lege eine gerichtete 
		% und mit $X$ bezeichnete Kante $T_j\underset{X}{\longrightarrow} T_k$ 	
		Kante $T_j\longrightarrow T_k$ 
		im Graphen an, wenn $O_{j}(X)$ vor $O_{k}(X)$ in $S$ auftritt.
		\pause
		\item Der Ausführungsplan $S(T_1,T_2,\dots ,T_n)$ ist serialisierbar, genau dann wenn der so entstandene Graph keine 
		gerichteten Zykel hat.
		% dessen Kanten alle den gleichen Bezeichner $X$ haben.
	\end{enumerate}
\end{block}
\end{frame}

\begin{frame}{\insertsection}
\framesubtitle{\insertsubsection}
\textbf{Beispiel: Serialisierbarkeitstest}
\begin{columns}
	\begin{column}{.65\textwidth}
		\begin{block}{}
			{\small
				Ausführungsplan $S$ der Transaktionen $T_1\,,\ldots ,\!T_4$:
				\begin{equation*}
				\begin{split}
				T_1 &= (r_1(X)\text{-}w_1(Y)\text{-}w_1(X))\\
				T_2 &= (w_2(Y)\text{-}r_2(X))\\
				T_3 &= (r_3(Y))\\
				T_4 &= (r_4(Y)\text{-}w_4(Y))\\
				\\[-8pt]
				S &= (r_1(X)\text{-}w_2(Y)\text{-}r_3(Y)\text{-}r_2(X)\text{-}w_1(Y)\text{-}r_4(Y)\text{-}w_1(X)\text{-}w_4(Y))
				\end{split}
				\end{equation*}
			}
		\end{block}
	\end{column}
	\begin{column}{.3\textwidth}
		\begin{figure}
			\includegraphics<2>[scale=0.6]{img/serializable-test1.png}
			\includegraphics<3>[scale=0.6]{img/serializable-test2.png}
			\includegraphics<4>[scale=0.6]{img/serializable-test3.png}
			\includegraphics<5>[scale=0.6]{img/serializable-test.png}
		\end{figure}
	\end{column}
\end{columns}
\abs\ \abs
\pause[5]
Keine gerichteten Zykel $\Rightarrow S$ ist serialisierbar
\end{frame}

\begin{frame}{\insertsection}
\framesubtitle{\insertsubsection}
\begin{block}{\textbf{\"Ubungsaufgabe}}
	Pr\"ufen Sie mit dem Serialisierbarkeitstest, welcher der beiden folgenden Ausführungspl\"ane serialisierbar ist.
	\begin{equation*}
	\begin{split}
	S_\Romannum{1} &= (r_1(X) - r_2(X) - w_1(X) - w_2(X))\\
	S_\Romannum{2} &= (r_1(X) - w_1(X) - r_2(X) - w_2(X) - r_2(Y) - w_1(Y))
	\end{split}
	\end{equation*}
\end{block}
\end{frame}

%\subsection{Wiederherstellbare Ausführungspläne}
%\begin{frame}
%\frametitle{Recovery}
%\framesubtitle{Wiederherstellbare Ausführungspläne}
%\structure{Wann kann eine Transaktion wieder rückgängig gemacht werden?}
%\begin{itemize}
%\item Zurücksetzen von Transaktionen erfolgt grundsätzlich nur bei %Transaktionen, die im Database Log \textit{keinen} Commit-Eintrag haben.
%\item Vor dem Commit-Eintrag im Database Log werden alle noch nicht persistierten Daten auf Platte geschrieben. 
%\item Ist der Commit Point erreicht, wird eine Transaktion nicht mehr rückgängig gemacht.
%\item Der Ausführungsplan muss ein Recovery zulassen: 
%\begin{itemize}
%\item Wiederherstellbare Ausführungspläne 
%\item Nicht wiederherstellbare Ausführungspläne
%\end{itemize}
%\end{itemize}
%\end{frame}

%\begin{frame}
%\frametitle{Recovery}
%\framesubtitle{Wiederherstellbare Ausführungspläne}
%\begin{definition}[Wiederherstellbare Ausführungspläne]
%Ein Ausführungsplan $S$ ist wiederherstellbar, wenn keine Transaktion $T \in S$ ein Commit ausgibt, bis alle Transaktionen $T'$, die ein von $T$ gelesenes Objekt geschrieben haben, erfolgreich beendet wurden.
%\end{definition}
%\begin{block}{Beispiele}
%\begin{itemize}
%\item $S_a:r_1(X); r_2(X); w_1(X); r_1(Y); w_2(X); c_2; w_1(Y); c_1$ ist %wiederherstellbar
%\item $S_b: r_1(X); w_1(X); r_2(X); r_1(Y); w_2(X); c_2; a_1$ ist nicht %wiederherstellbar
%\end{itemize}
%\alert{In $S_b$ liest $T_2$ den wert $X$, der zuvor von $T_1$ geschrieben wurde. %$T_2$ gibt aber vor $T_1$ das Commit aus. Daher ist der Ausführungsplan nicht wiederherstellbar.}
%\end{block}
%\end{frame}

%\begin{frame}
%\frametitle{Recovery}
%\framesubtitle{Wiederherstellbare Ausführungspläne}
%\begin{block}{Kaskadierendes Rollback}
%$$S_c: r_1(X); w_1(X); r_2(X); r_1(Y); w_2(X); w_1(Y); a_1; a_2$$
%Eine noch nicht bestätigte Transaktion (hier: $T_2$) muss wieder rückgängig gemacht werden, wenn Sie Werte aus einer anderen Transaktion (hier: $T_1$) gelesen hat, die dann abgebrochen wird. Der Abbruch der Transaktion zieht dann weitere Transaktionsabbrüche nach sich -- man spricht von \textit{kaskadierten} Abbrüchen.
%\end{block}
%\begin{definition}[Kaskadenfreier Ausführungsplan]
%Ein Ausführungsplan $S$ ist kaskadenfrei, wenn alle $T_i \in S$ nur dann Objekte %aus $T_j \in S$ lesen, wenn $T_j$ bereits ein Commit geschickt hat.
%\end{definition}
%\end{frame}

%\begin{frame}
%\frametitle{Recovery}
%\framesubtitle{Wiederherstellbare Ausführungspläne}
%\structure{Strategien des Recovery Managers:}
%\begin{itemize}
%	\item Commits können verzögert werden, so dass eine Wiederherstellbarkeit des Ausführungsplanes erzwungen wird
%	\item Ausführungspläne werden nach Möglichkeit kaskadenfrei erzeugt
%\end{itemize}
%\alert{Aber nicht alle Ausführungspläne können Recovery-Fähig gestaltet werden:}
%$$S: r_1(X); w_1(X); r_2(Y); w_2(Y); r_1(Y); w_1(Y); r_2(X); w_2(X); c_1; c_2$$
%\alert{Weder $T_1$ noch $T_2$ darf als erstes ein Commit schicken. Für diese Fälle müssen übergeordnete Mechanismen zur Verfügung stehen.}
%\end{frame}

%
% @@@ hier weiter @@@
%

\section{Nebenläufigkeit}
\subsection{Sperrprotokolle}

\begin{frame}{\insertsection}
	\framesubtitle{\insertsubsection}
\structure{Sperrprotokolle bilden eines der wichtigesten Konzepte zur Nebenläufigkeitskontrolle:}
\begin{itemize}
\item Gängigste Vertreter: Zwei-Phasen-Sperrprotokolle (2PL)
\item Basieren auf dem \textit{Lock}-Konzept:
\begin{itemize}
\item Binäre Sperren
\item Gemeinsame und exklusive Sperren 
\end{itemize}
\item Kritische Abschnitte werden vor dem Zugriff durch andere Transaktionen geschützt
\end{itemize}
\structure{Datenbank-Operationen:}
\begin{itemize}
\item \texttt{lock\_item(X)} sperrt ein Objekt X
\item \texttt{unlock\_item(X)} hebt die Sperre für X wieder auf 
\item \texttt{lock(X)} prüft, ob X mit einer Sperre belegt ist
\end{itemize}
\end{frame}

\begin{frame}{\insertsection}
	\framesubtitle{\insertsubsection}
\structure{Anwendung in DBMS:}
\begin{itemize}
\item Im Allgemeinen sind binäre Locks nicht für den Einsatz in DBMS geeignet, das sie zu restriktiv sind und damit den Grad der Nebenläufigkeit zu start einschränken.
\item Konzept der exklusiven bzw. gemeinsamen Sperren (shared und exklusive locks) hier besser geeignet. 
\item Operationen:
\begin{itemize}
\item \texttt{read\_lock(X)} setzt ein gemeinsames (shared) Lock auf Objekt X
\item \texttt{write\_lock(X)} setzt ein exklusives Lock auf Objekt X 
\item \texttt{unlock(X)} löst ein gemeinsames oder exklusives Lock auf Objekt X 
\end{itemize} 
\end{itemize}
\end{frame}

\begin{frame}{\insertsection}
	\framesubtitle{\insertsubsection}
\structure{Vorgehen:}
\begin{enumerate}
\item Eine Transaktion $T$ muss die Operation \texttt{read\_lock(X)} oder \texttt{write\_lock(X)} ausführen, bevor eine \texttt{read\_item(X)}-Operation in $T$ ausgeführt wird
\item Eine Transaktion $T$ muss die Operation \texttt{write\_lock(X)} ausführen, bevor eine \texttt{write\_item(X)}-Operation in $T$ ausgeführt wird
\item Eine Transaktion $T$ muss die Operation \texttt{unlock(X)} ausführen, nachdem alle Operationen \texttt{read\_item(X)} und \texttt{write\_item(X)} in $T$ abgeschlossen sind
\item Eine Transaktion $T$ gibt keine \texttt{read\_lock(X)}-Operation aus, wenn sie schon ein Lock für X hält (read- oder write lock)
\item Eine Transaktion $T$ gibt keine \texttt{write\_lock(X)}-Operation aus, wenn sie schon ein Lock für X hält (read- oder write lock)
\item Eine Transaktion $T$ gibt nur dann ein \texttt{unlock(X)} aus, wenn sie bereits ein Lock auf X hält
\end{enumerate}
\end{frame}

\begin{frame}{\insertsection}
	\framesubtitle{\insertsubsection}
	\begin{definition}[Deadlock / Verklemmung]
		Ein Deadlock entsteht, wenn jede Transaktion $T$ in einer Menge von zwei oder mehr Transaktionen auf ein Element wartet, welches vorher von einer anderen Transaktion $T'$ in der selben Menge gesperrt wurde.
	\end{definition}	
	\begin{columns}
		\begin{column}{.48\textwidth}
			\begin{tabular}{p{3.0cm}|p{3.0cm}}
				\multicolumn{1}{c|}{\small $T_1$} & \multicolumn{1}{c}{\small $T_2$}\\\hline
				\small \texttt{read\_lock(Y)} & \\
				\small \texttt{read\_item(Y)} & \\
				& \small \texttt{read\_lock(X)} \\
				& \small \texttt{read\_item(X)} \\
				\small \texttt{write\_lock(X)} & \\
				& \small \texttt{write\_lock(Y)} \\
			\end{tabular}
		\end{column}
		\begin{column}{.48\textwidth}			
			\begin{center}
				\begin{tikzpicture}
				\node[draw, circle] (t1) {$T_1$}; 
				\node[draw, circle] (t2) [right= 2.5cm of t1] {$T_2$};
				\node[draw=none, ] (text) [right=0mm of t1]{\small Wait-For-Graph};
				\draw (t1.north)  edge[out=90, in = 90, ->](t2.north); 
				\draw (t2.south)  edge[out=270, in = 270, ->](t1.south);  
				\end{tikzpicture}
			\end{center}
		\end{column}
	\end{columns}
\end{frame}

\begin{frame}{\insertsection}
	\framesubtitle{\insertsubsection}
\begin{definition}[Basic 2PL]
Eine Transaktion $T$ folgt dem Zwei-Phasen-Sperrprotokoll genau dann, wenn alle Lock-Operationen \textit{vor} der ersten Unlock-Operation durchgeführt werden.
\end{definition}

\structure{Damit wird das 2PL in die folgenden zwei Phasen unterteilt:}
\begin{enumerate}
\item Expand- bzw. Grow-Phase: 
\begin{itemize}
\item Neue Locks können angefordert werden 
\item Keine Locks werden abgebaut 
\end{itemize}
\item Release- bzw. Shrink-Phase: 
\begin{itemize}
\item Bestehende Locks werden abgebaut 
\item Es werden keine neuen Locks angefordert
\end{itemize}
\end{enumerate}
\end{frame}

\begin{frame}{\insertsection}
	\framesubtitle{\insertsubsection}
\begin{definition}[Conservative 2PL]
Eine Transaktion $T$ folgt dem konservativen Zwei-Phasen-Sperrprotokoll genau dann, wenn alle Locks \textit{vor} der eigentlichen Transaktionsausführung angefordert werden.
\end{definition}
\begin{itemize}
\item Die Transaktion muss ihre sogenannten \textit{Read Sets} und \textit{Write Sets}, d.h. alle in ihr vorkommenden Objekte, die gelesen oder geschrieben werden, vordeklarieren. 
\item Wenn ein Element nicht gesperrt werden kann, wird keines der Elemente gesperrt 
\item Die Transaktion wartet dann solange, bis alle Elemente mit einem Lock belegt werden können
\item Conservative 2PL ist Deadlock-Frei 
\item \alert{Nicht immer können im laufenden Betrieb die Read- und Write Sets bestimmt werden, bevor die Transaktion ausgeführt wird.}
\end{itemize}
\end{frame}

\begin{frame}{\insertsection}
	\framesubtitle{\insertsubsection}
\begin{definition}[Strict 2PL]
Eine Transaktion $T$ gibt kein Write-Lock frei, bis sie 
\begin{itemize}
\item entweder abgebrochen wird oder 
\item in einen \texttt{committed}-Zustand wechselt.
\end{itemize}
\end{definition}

\begin{definition}[Rigorous 2PL]
Eine Transaktion $T$ gibt kein Read- oder Write-Lock frei, bis sie 
\begin{itemize}
\item entweder abgebrochen wird oder 
\item in einen \texttt{committed}-Zustand wechselt.
\end{itemize}
\end{definition}
\alert{Während Conservative 2PL alle Elemente vor Beginn sperrt, gibt Rigorous 2PL keine Elemente vor Terminierung frei. Damit ist Rigorous 2PL bis zur Terminierung in der Expand-Phase.}
\end{frame}

\subsection{Deadlock-Vermeidung und Deadlock-Verhinderung}

\begin{frame}{\insertsection}
	\framesubtitle{\insertsubsection}
\structure{Es existieren eine Reihe von Protokollen zur Deadlock-Vermeidung:}
\begin{itemize}
\item Conservative 2PL setzt alle Locks im Voraus und wartet ansonsten so lange, bis alles Locks verfügbar sind. Dies geht auf Kosten der Parallelität.
\item Zeitstempelbasierte Verfahren: Eine Transaktion $T_1$ erhält einen Zeitstempel $TS(T_1)$. 
\begin{itemize}
\item \textit{Wait-Die-Verfahren:} Wenn $TS(T_1)<TS(T_2)$, dann darf $T_1$ auf Lock-Freigabe warten, andernfalls wird $T_1$ abgebrochen. Die ältere Transaktion wartet auf die jüngere.
\item \textit{Wound-Wait-Verfahren:} Wenn $TS(T_1)<TS(T_2)$, dann wird $T_2$ abgebrochen ($T_1$ verwundet $T_2$) -- in diesem Fall wartet die jüngere Transaktion auf die ältere.
\end{itemize}
\item Verfahren ohne Zeitstempel: 
\begin{itemize}
\item \textit{No-Waiting-Algorithmus}: Wenn eine Transaktion ein Lock nicht bekommt, wird sie sofort abgebrochen
\item \textit{Cautious-Waiting-Algorithmus:} Eine Transaktion darf nur dann in einen wartenden Zustand verfallen, wenn die jeweils andere Transaktion nicht in einem wartenden Zustand ist. 
\end{itemize}
\end{itemize}
\alert{Alle Verfahren sind Deadlock-Frei, aber es werden mitunter Transaktionen unnötig abgebrochen. $\Rightarrow$ Deadlock Detection} 
\end{frame}

\subsection{Deadlock Detection}
\begin{frame}{\insertsection}
	\framesubtitle{\insertsubsection}
\structure{Wait-For-Graph-basierte Ansätze}
\begin{itemize}
\item Erzeuge für jede Transaktion einen Knoten im Graphen
\item Eine Transaktion, die auf eine andere wartet, wird durch eine gerichtete Kante dargestellt. 
\item Ein Deadlock entsteht genau dann, wenn der Graph einen Zyklus enthält. 
\item \alert{Problematisch: Wann soll eine Prüfung durchgeführt werden und welche Transaktion soll abgebrochen werden?}
\end{itemize}
\structure{Timeouts}
\begin{itemize}
\item Die Vermutung eines Deadlocks entsteht genau dann, wenn eine Transaktion länger als eine vorgegebene Zeit wartet. 
\item Einfacher Ansatz, aber: 
\item \alert{Wie lange soll das Zeitfenster sein?}
\end{itemize}
\end{frame}

\subsection{Isolationsstufen in SQL}

\begin{frame}{\insertsection}
	\framesubtitle{\insertsubsection}
\structure{Mögliche Stufen:}
\begin{itemize}
\item READ UNCOMMITTED (Dirty Read, Lost Update, Phantom Read, Non-Repeatable Read)
\begin{itemize}
	\item Lese-Transaktionen können ohne Sperren lesen
\end{itemize}
\item READ COMMITTED (Lost Update, Phantom Read, Non-Repeatable Read)
\begin{itemize}
	\item Transaktionen lesen nur Werte, die festgeschrieben sind (Write Locks mit Strict 2PL)
\end{itemize}
\item READ STABILITY (Phantom Read)
\begin{itemize}
	\item Schreib- und Lese-Locks werden mit Strict 2PL angefordert, Phantom-Problem kann existieren
\end{itemize}
\item SERIALIZABLE
\begin{itemize}
	\item Ausschließliche Ausführung serialisierbarer Transaktionen
\end{itemize}
\end{itemize}
\end{frame}

\section*{Übungsaufgaben}

\begin{frame}[t]
	\frametitle{\insertsection}	
	\begin{alertblock}{Transaktionen}
		\begin{enumerate}
			\item Definieren Sie den Begriff \textit{Transaktion}. Wie werden in SQL Transaktionen eingeleitet bzw. abgeschlossen?
			\item Klassifizieren Sie die Fehlerquellen bei Transaktionen nach ihrer Eintrittswahrscheinlichkeit.
			\item Erläutern Sie die unterschiedlichen Varianten der Zwei-Phasen-Sperrprotokolle.
			\item Definieren Sie den Begriff \textit{Deadlock}. Wie kann ein Deadlock mittels eines Wait-For-Graphen identifiziert werden?
			\item Experimentieren Sie mit dem RDBMS: 
			\begin{enumerate}
				\item Versuchen Sie, einen Deadlock zu provozieren.
				\item Erläutern Sie das Verhalten eines RDBMS, wenn Transaktionen im Isolationslevel \texttt{READ COMMITTED} bzw. \texttt{READ UNCOMMITTED} arbeiten. Erstellen Sie ein Beispiel!
			\end{enumerate}		
		\end{enumerate}
	\end{alertblock}
\end{frame}

\begin{frame}[t]
	\frametitle{\insertsection}	
	\begin{alertblock}{Serialisierbarkeit}
		Gegeben sind die folgenden Ausführungspläne: 
		\begin{itemize}
			\item[$S_1$:] $r_1(X)\text{-} r_2(Y)\text{-} r_3(X)\text{-} w_3(X)\text{-} r_1(Y)\text{-} w_1(Y)\text{-} r_2(Z)\text{-} r_1(Z)\text{-} r_2(W)\text{-} r_1(W)\text{-} w_1(W)\text{-} w_3(Z)$
			\item[$S_2$:] $w_1(X)\text{-} r_2(Y)\text{-} r_2(X) \text{-}w_3(Y)\text{-} r_3(Z)\text{-} r_4(P)\text{-} w_4(Z)\text{-} w_1(P)\text{-} r_1(Q)\text{-} w_3(Q)$
			\item[$S_3$:] $r_1(X)\text{-} r_2(Z)\text{-} r_3(X)\text{-} r_1(Z)\text{-} r_2(Y)\text{-} r_3(Y)\text{-} w_1(X)\text{-} w_2(Z)\text{-} w_3(Y)\text{-} w_2(Y)$		
		\end{itemize}
	Erstellen Sie die Konfliktgraphen f\"ur diese Ausf\"uhrungspl\"ane. 
	\nl
	Welche Ausf\"uhrungspl\"ane sind serialisierbar?					
	\end{alertblock}
\end{frame}
