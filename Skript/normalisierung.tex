%!TEX root = Slides.tex
\part{Entwurfstheorie}

\section{Problemstellung}

\begin{frame}\frametitle{\insertsection}
\begin{itemize}
	\item Bisher: Top-Down-Entwurf: Von der Anforderungsanalyse \"uber das ERM zum relationalen Schema
	\begin{itemize}
		\item Oft Datenredundanz, hohe Anzahl Attribute in Tabellen, etc. in Kauf genommen
		\item Oder zu viele schmale Relationen, Entit\"aten und Einheiten auseinandergebrochen, etc.
	\end{itemize}
  \pause
  \abs
	\item Besser: Konzeptuelle Feinabstimmung der erstellten relationalen Schemata
	\begin{itemize}
		\item Grundlage: Funktionale Abh\"angigkeiten und Schl\"ussel
		\item Ziel: G\"utebestimmung und Anpassung der Relationenschemata mit Hilfe von Normalformen und Normalisierungsalgorithmen
	\end{itemize}
	\end{itemize}
\end{frame}

\subsection{Breite Relationen}

\begin{frame}[t]\frametitle{\insertsection}
	\framesubtitle{\insertsubsection}
 	\structure{Gegeben sei die folgende recht breite Relation:}
 	\begin{center}
		\begin{tabular}{|c|c|c|c|c|c|}\hline
			\multicolumn{6}{|c|}{\small \textbf{Mitarbeiter\_Projekt}}\\\hline\hline
			 \small \textbf{\key{MANr}} & \small \textbf{Vorname}&\small \textbf{Nachname}&\textbf{\key{PNr}} &\small \textbf{Projektname}&\small \textbf{Funktion} \\\hline 
			\small 1 &\small Max & \small Mustermann &\small 4711 &\small Entwicklung &\small Entwickler \\\hline 
			\small 1 &\small Max & \small Mustermann &\small 4713 &\small Vertrieb & \small Kampagnen \\\hline 
			\small 2 &\small Erika &\small Musterfrau &\small 4711 &\small Entwicklung &\small SCRUM Master \\\hline 
			\small 2 &\small Erika &\small Musterfrau &\small 4712 &\small Test &\small Integrations-Tests \\\hline 
			\small 3 &\small Moritz & \small Musterherr &\small 4713 &\small Research &\small Leitung \\\hline 
		\end{tabular}
	\end{center}
	Relation hat einige problematische Eigenschaften, die zu \textit{Anomalien} führen können. 
\end{frame}

\begin{frame}[t]\frametitle{\insertsection}
\framesubtitle{\insertsubsection}
\structure{Redundanzproblem}
\begin{center}
	\begin{tabular}{|c|c|c|c|c|c|}\hline
		\multicolumn{6}{|c|}{\small \textbf{Mitarbeiter\_Projekt}}\\\hline\hline
		\small\textbf{\key{MANr}}&\small\textbf{Vorname}&\small\textbf{Nachname}&\textbf{\key{PNr}}&\small\textbf{Projektname}&\small\textbf{Funktion}\\\hline 
		\small 1 &\small Max & \small Mustermann &\small 4711 &\small Entwicklung &\small Entwickler \\\hline 
		\small \cellcolor{Red}1 &\small \cellcolor{Red}Max & \small \cellcolor{Red}Mustermann &\small 4713 
		&\small Vertrieb & \small Kampagnen \\\hline 
		\small 2 &\small Erika &\small Musterfrau &\small \cellcolor{Red}4711 &\small \cellcolor{Red}Entwicklung &\small SCRUM Master \\\hline 
		\small \cellcolor{Red}2 &\small \cellcolor{Red}Erika &\small \cellcolor{Red}Musterfrau 
		&\small 4712 &\small Test &\small Integrations-Tests \\\hline 
		\small 3 &\small Moritz & \small Musterherr &\small 4714 &\small Research &\small Leitung \\\hline 
	\end{tabular}
\end{center} 	
Datenredundanz: Werte müssen ggf.~mehrfach eingegeben werden
\end{frame}

\begin{frame}[t]\frametitle{\insertsection}
\framesubtitle{\insertsubsection}
\structure{Einf\"ugeanomalien}
\begin{center}
\begin{tabular}{|c|c|c|c|c|c|}\hline
	\multicolumn{6}{|c|}{\small \textbf{Mitarbeiter\_Projekt}}\\\hline\hline
	\small\textbf{\key{MANr}}&\small\textbf{Vorname}&\small\textbf{Nachname}&\textbf{\key{PNr}}&\small\textbf{Projektname}&\small\textbf{Funktion}\\\hline 
	\small 1 &\small Max & \small Mustermann &\small 4711 &\small Entwicklung &\small Entwickler \\\hline 
	\small 1 &\small Max & \small Mustermann &\small 4713 &\small Vertrieb & \small Kampagnen \\\hline 
	\small \cellcolor{red}\texttt{null} &\small \cellcolor{red}\texttt{null} & \small \cellcolor{red}\texttt{null}
	&\small 4712 & \small Test &\small Unit-Tests \\\hline 		
	\small 2 &\small Erika &\small Musterfrau &\small 4711 &\small Entwicklung &\small SCRUM Master \\\hline 
	\small 2 &\small Erika &\small Musterfrau &\small 4712 &\small Test &\small Integrations-Tests \\\hline 
	\small 3 &\small Moritz & \small Musterherr &\small 4714 &\small Research &\small Leitung \\\hline 
\end{tabular}
\end{center} 	
\begin{itemize}
\item Neues Projekt und neue Funktion ohne Zuordnung zu Mitarbeiter nicht möglich $\Rightarrow$ Primary Key Coinstraint
\item Entsprechend: Neue Mitarbeiterin ohne Projekt-/Funktionszuordnung nicht möglich.
\end{itemize}
\end{frame}

\begin{frame}[t]\frametitle{\insertsection}
\framesubtitle{\insertsubsection}
\structure{Einf\"ugeanomalien}
\begin{center}
	\begin{tabular}{|c|c|c|c|c|c|}\hline
		\multicolumn{6}{|c|}{\small \textbf{Mitarbeiter\_Projekt}}\\\hline\hline
		\small\textbf{\key{MANr}}&\small\textbf{Vorname}&\small\textbf{Nachname}&\textbf{\key{PNr}}&\small\textbf{Projektname}&\small\textbf{Funktion}\\\hline 
		\small 1 &\small Max & \small Mustermann &\small 4711 &\small Entwicklung &\small Entwickler \\\hline 
		\small 1 &\small Max & \small Mustermann &\small 4713 &\small Vertrieb & \small Kampagnen \\\hline 
		\small 1 &\small Max & \small \cellcolor{red}Musterrann &\small 4712 & \small Test &\small Unit-Tests \\\hline 		
		\small 2 &\small Erika &\small Musterfrau &\small 4711 &\small Entwicklung &\small SCRUM Master \\\hline 
		\small 2 &\small Erika &\small Musterfrau &\small 4712 &\small Test &\small Integrations-Tests \\\hline 
		\small 3 &\small Moritz & \small Musterherr &\small 4714 &\small Research &\small Leitung \\\hline 
	\end{tabular}
\end{center} 	
\begin{itemize}
	\item Wegen Redundanz h\"ohere Wahrscheinlichkeit von Dateninkonsistenz, bspwse.~durch Tippfehler.
\end{itemize}
\end{frame}

\begin{frame}[t]\frametitle{\insertsection}
\framesubtitle{\insertsubsection}
\structure{Einf\"ugeanomalien}
\begin{center}
\begin{tabular}{|c|c|c|c|c|c|}\hline
\multicolumn{6}{|c|}{\small \textbf{Mitarbeiter\_Projekt}}\\\hline\hline
\small\textbf{\key{MANr}}&\small\textbf{Vorname}&\small\textbf{Nachname}&\textbf{\key{PNr}}&\small\textbf{Projektname}&\small\textbf{Funktion}\\\hline 
\small 1 &\small Max & \small Mustermann &\small 4711 &\small Entwicklung &\small Entwickler \\\hline 
\small 1 &\small Max & \small Mustermann &\small 4713 &\small Vertrieb & \small Kampagnen \\\hline 
\small 1 &\small Max & \small Mustermann &\small 4712 & \small Test &\small Unit-Tests \\\hline 		
\small \cellcolor{red}1 &\small Max &\small Mustermann &\small \cellcolor{red}4712 &\small Test &\small Integrations-Tests \\\hline 
\small 2 &\small Erika &\small Musterfrau &\small 4711 &\small Entwicklung &\small SCRUM Master \\\hline 
\small 2 &\small Erika &\small Musterfrau &\small 4712 &\small Test &\small Integrations-Tests \\\hline 
\small 3 &\small Moritz & \small Musterherr &\small 4714 &\small Research &\small Leitung \\\hline 
\end{tabular}
\end{center} 	
\begin{itemize}
\item Funktionserweiterung f\"ur Mitarbeiter im Projekt nicht m\"oglich $\Rightarrow$ Primary Key Coinstraint
\end{itemize}
\end{frame}

\begin{frame}[t]\frametitle{\insertsection}
\framesubtitle{\insertsubsection}
\structure{L\"oschanomalien}
\begin{center}
\begin{tabular}{|c|c|c|c|c|c|}\hline
\multicolumn{6}{|c|}{\small \textbf{Mitarbeiter\_Projekt}}\\\hline\hline
\small\textbf{\key{MANr}}&\small\textbf{Vorname}&\small\textbf{Nachname}&\textbf{\key{PNr}}&\small\textbf{Projektname}&\small\textbf{Funktion}\\\hline 
\small 1 &\small Max & \small Mustermann &\small 4711 &\small Entwicklung &\small Entwickler \\\hline 
\small 1 &\small Max & \small Mustermann &\small 4713 &\small Vertrieb & \small Kampagnen \\\hline 
\small 1 &\small Max & \small Mustermann &\small 4712 & \small Test &\small Unit-Tests \\\hline 		
\small 2 &\small Erika &\small Musterfrau &\small 4711 &\small Entwicklung &\small SCRUM Master \\\hline 
\small 2 &\small Erika &\small Musterfrau &\small 4712 &\small Test &\small Integrations-Tests \\\hline 
\small 3 &\small\sout{Moritz} & \small \sout{Musterherr} 
&\small \cellcolor{Red}\sout{4714} &\small \cellcolor{Red}\sout{Research} &\small \cellcolor{Red}\sout{Leitung} \\\hline 
\end{tabular}
\end{center} 	
\begin{itemize}
\item L\"oschen eines Datensatzes vernichtet Informationen \"uber andere Sachverhalte/Entit\"aten.
\item Beispiel: L\"oschen des Mitarbeiters in Projekt 4714 bedingt L\"oschen des Projektes.
\end{itemize}
\end{frame}

\begin{frame}[t]\frametitle{\insertsection}
	\framesubtitle{\insertsubsection}
 	\structure{Modifikationsanomalien}
 	\begin{center}
 		\begin{tabular}{|c|c|c|c|c|c|}\hline
 		 \multicolumn{6}{|c|}{\small \textbf{Mitarbeiter\_Projekt}}\\\hline\hline
 		 \small\textbf{\key{MANr}}&\small\textbf{Vorname}&\small\textbf{Nachname}&\textbf{\key{PNr}}&\small\textbf{Projektname}&\small\textbf{Funktion}\\\hline 
 		 \small 1 &\small Max & \small Mustermann &\small 4711 &\small Entwicklung &\small Entwickler \\\hline 
 		 \small 1 &\small Max & \small Mustermann &\small 4713 &\small Vertrieb & \small Kampagnen \\\hline 
 		 \small 1 &\small Max & \small Mustermann &\small 4712 & \small Test &\small Unit-Tests \\\hline 		
 		 \small 2 &\small Erika &\small Musterfrau &\small 4711 &\small Entwicklung &\small SCRUM Master \\\hline 
 		 \small 2 &\small Erika &\small \cellcolor{red}Mustermann &\small 4712 &\small Test &\small Integrations-Tests \\\hline 
 	  \end{tabular}
	\end{center}
	\begin{itemize}
		\item \"Anderungen eines Attributwertes (z.B. Nachname von Frau Musterfrau) m\"ussen in allen Tupeln durchgef\"uhrt werden
		\item Werden Tupel vergessen, entstehen inkonsistente Datenbest\"ande
	\end{itemize}
\end{frame}

\begin{frame}[t]\frametitle{\insertsection}
	\framesubtitle{\insertsubsection}
 	\structure{Die Relation nach ein paar Tagen im operativen Einsatz:}
 	\begin{center}
		\begin{tabular}{|c|c|c|c|c|c|}\hline
			\multicolumn{6}{|c|}{\small \textbf{Mitarbeiter\_Projekt}}\\\hline\hline
			 \small \textbf{\key{MANr}} & \small \textbf{Vorname}&\small \textbf{Nachname}&\small\textbf{\key{PNr}} &\small \textbf{Projektname}&\small \textbf{Funktion} \\\hline 
			\small 1 &\small Max &\small Mustermann &\small 4711 & \small Entwicklung &\small Entwickler \\\hline 
			\small 1 &\small Max &\small \cellcolor{Red}Mustrmann &\small 4712 &\small Test &\small Unit-Tests \\\hline 
			\small 2 &\small Erika &\small Musterfrau &\small 4711 & Entwicklung &\small SCRUM Master \\\hline 
			\small 2 &\small Erika &\small \cellcolor{Red}Mustermann &\small 4712 &\small Test &\small Integrations-Tests \\\hline 
			\cellcolor{Red}\small\sout{3} &\cellcolor{Red}\small\sout{Moritz}&\cellcolor{Red}\small\sout{Musterherr}
			  &\cellcolor{Red}\small\sout{4714}&\cellcolor{Red}\small\sout{Research} & \cellcolor{Red}\small\sout{Leitung}\\\hline
			\cellcolor{Yellow}\small 9999 &\cellcolor{Yellow}\small \texttt{null} &\cellcolor{Yellow} \texttt{null}
			   &\cellcolor{Yellow}\small 4715 & \cellcolor{Yellow}\small App-Development  &\small \cellcolor{Yellow} Dokumentation  \\\hline 
			\cellcolor{Yellow}\small 4 &\cellcolor{Yellow}\small Michael &\cellcolor{Yellow}\small Franke  
			& \cellcolor{Yellow}\small 9999 & \cellcolor{Yellow}\small \texttt{null}  &\small \cellcolor{Yellow} \texttt{null}  \\\hline 
		\end{tabular}
	\end{center}	
\end{frame}

\subsection{Schmale Relationen}

\begin{frame}[t]\frametitle{\insertsection}
\framesubtitle{\insertsubsection}
\structure{Schema f\"ur Mitarbeiter- und Projektdaten in Form von schmalen Relationen:}
\begin{center}
	\begin{tabular}{|c|c|c|c|}\hline
		\multicolumn{4}{|c|}{\small \textbf{Mitarbeiter}}\\\hline\hline
		\small \textbf{\key{MANr}} & \small \textbf{Vorname}&\small \textbf{Nachname}&\textbf{\key{PNr}}\\\hline 
		\small 1 &\small Max & \small Mustermann &\small 4711\\\hline 
		\small 1 &\small Max & \small Mustermann &\small 4713\\\hline 
		\small 2 &\small Erika &\small Musterfrau &\small 4711\\\hline 
		\small 2 &\small Erika &\small Musterfrau &\small 4712\\\hline 
		\small 3 &\small Moritz & \small Musterherr &\small 4713\\\hline 
	\end{tabular}
	\hspace{4mm}
	\begin{tabular}{|c|c|c|}\hline
		\multicolumn{3}{|c|}{\small \textbf{Projekt}}\\\hline\hline
		\textbf{PNr} &\small \textbf{Projektname}&\small \textbf{Funktion} \\\hline 
		\small 4711 &\small Entwicklung &\small Entwickler \\\hline 
		\small 4713 &\small Vertrieb & \small Kampagnen \\\hline 
		\small 4711 &\small Entwicklung &\small SCRUM Master \\\hline 
		\small 4712 &\small Test &\small Integrations-Tests \\\hline 
		\small 4713 &\small Research &\small Leitung \\\hline 
	\end{tabular}
\end{center}
\begin{itemize}
	\item Zusammenf\"uhrung (Join) der Einheiten kann zu unerw\"unschten Datens\"atzen f\"uhren.
\end{itemize}
\end{frame}

\begin{frame}[t]\frametitle{\insertsection}
\framesubtitle{\insertsubsection}
\structure{Weiteres Beispiel: Schema f\"ur Mitarbeiterdaten mit vielen schmalen Relationen}
\begin{center}
	\begin{tabular}{|c|c|}\hline
		\multicolumn{2}{|c|}{\small \textbf{MA\_Vorname}}\\\hline\hline
		\small \textbf{\key{MANr}} & \small \textbf{Vorname}\\\hline 
		\small 1 &\small Max \\\hline 
		\small 2 &\small Erika \\\hline 
	\end{tabular}
	\hspace{4mm}
	\begin{tabular}{|c|c|}\hline
		\multicolumn{2}{|c|}{\small \textbf{MA\_Nachname}}\\\hline\hline
		\small \textbf{\key{MANr}} & \small \textbf{Nachname}\\\hline 
		\small 1 &\small Mustermann \\\hline 
		\small 2 &\small Musterfrau \\\hline 
	\end{tabular}
	\abs
	\begin{tabular}{|c|c|}\hline
		\multicolumn{2}{|c|}{\small \textbf{MA\_PLZ}}\\\hline\hline
		\small \textbf{\key{MANr}} & \small \textbf{PLZ}\\\hline 
		\small 1 &\small 68165 \\\hline 
		\small 2 &\small 76243 \\\hline 
	\end{tabular}
	\hspace{4mm}
	\begin{tabular}{|c|c|}\hline
		\multicolumn{2}{|c|}{\small \textbf{MA\_Ort}}\\\hline\hline
		\small \textbf{\key{MANr}} & \small \textbf{Ort}\\\hline 
		\small 1 &\small Mannheim \\\hline 
		\small 2 &\small Karlsruhe \\\hline 
	\end{tabular}
	\hspace{4mm}
	\begin{tabular}{|c|c|}\hline
		\multicolumn{2}{|c|}{\small \textbf{MA\_Str}}\\\hline\hline
		\small \textbf{\key{MANr}} & \small \textbf{Strasse}\\\hline 
		\small 1 &\small Gartenstrasse \\\hline 
		\small 2 &\small Hauptstrasse \\\hline 
	\end{tabular}
\end{center}
\begin{itemize}
	\item Objekt-Semantik auf mehrere Tabellen verteilt; zu viele Abh\"angigkeiten untereinander 
	\item Hoher Aufwand, semantisch vollständige Daten zu erhalten: 
	\nl
	\texttt{MA\_Vorname} $\Join$ \texttt{MA\_Nachname}$\Join$ \texttt{MA\_PLZ} $\Join$ \texttt{MA\_Ort}$\Join$ \texttt{MA\_Str} 
\end{itemize}
\end{frame}

\subsection{Fazit}
\begin{frame}[t]\frametitle{\insertsection}
\framesubtitle{\insertsubsection}
\begin{block}{Zu breite Relationen}
	\begin{itemize}
		\item Sind abfragetechnisch einfach zu handhaben, leiden aber an Anomalien
		\item Vermischung unterschiedlicher Entit\"aten in einer Relation f\"uhrt zu Unsauberkeiten
	\end{itemize}
\end{block}
\pause
\begin{block}{Zu schmale Relationen}
	\begin{itemize}
		\item Unhandlich, da zu viele Relationen gepflegt werden m\"ussen
		\item Semantische Einheiten werden auseinander gebrochen und m\"ussen mit viel Aufwand wieder zusammengef\"ugt werden
		\item Zusammenf\"uhrung der Einheiten kann zu unerw\"unschten Datens\"atzen f\"uhren
	\end{itemize}
\end{block}
\pause
\begin{block}{\textbf{Ziel}}
	Finde Methodik und Systematik, um ein '\textit{gutes}' Datenmodell zu erstellen ...
\end{block}
\end{frame}

\section{Normalformen}

\subsection{Grundlagen}

\begin{frame}[t]
\frametitle{\insertsection}
\framesubtitle{\insertsubsection}
\begin{definition}[Normalisierung]
	Normalisierung ist ein Zerlegungsprozess eines Relationenschemas $\mcl{R}(\mcl{A})$ in mehrere Teilschemata 
	$\mcl{R}_1(\mcl{A}_1),\mcl{R}_2(\mcl{A}_2),\dots ,\mcl{R}_n(\mcl{A}_n)$ mit $\bigcup_{i=1}^n\mcl{A}_i = \mcl{A}$. 
	\abs
	F\"ur eine solche Zerlegung m\"ussen zwei grundlegende Korrektheitskriterien erf\"ullt sein:\\[4pt]
	\begin{enumerate}
		\pause
		\item \emph{Verlustlosigkeit}: Die Information einer beliebigen Relation $R$ des Schemas $\mcl{R}$ muss 
		eineindeutig aus den entsprechenden Relationen $R_1,\ldots , R_n$ der Schemata 
		$\mcl{R}_1,\dots ,\mcl{R}_n$ rekonstruierbar sein.\\[4pt]
		\pause
		\abs
		\item \emph{Abh\"angigkeitserhaltung}: Die funktionalen Abh\"angigkeiten des Schemas $\mcl{R}$ und die 
		funktionalen Abh\"angigkeiten der Schemata $\mcl{R}_1,\dots ,\mcl{R}_n$ m\"ussen aufeinander abbildbar sein.		
	\end{enumerate}
\end{definition}
\end{frame}
%
% Synonyme:
% Verlustlosigkeit          ~   Verbundtreue
% Abh\"angigkeitserhaltung  ~   Abh\"angigkeitstreue
%

\begin{frame}[t]
\frametitle{\insertsection}
\framesubtitle{\insertsubsection}
\onslide
\begin{block}{\textbf{Mit anderen Worten:}}
	\begin{itemize}
		\item \emph{Verlustlosigkeit}. F\"ur jede Relation $R$ und deren Zerlegung $R_1,R_2,\dots ,R_n$ muss $R$ mittels Join-Operation 
		aus $R_1,R_2,\dots ,R_n$ rekonstruiert werden k\"onnen.
		$$R=R_1\Join\ldots\Join R_n$$
		\pause
		\item \emph{Abh\"angigkeitserhaltung}. Die funktionalen Abh\"angigkeiten von $R$ m\"ussen in den Teilrelationen 
		$R_1,R_2,\dots , R_n$ enthalten sein. Es d\"urfen aus den Relationen $R_1,R_2,\dots, R_n$ keine weiteren funktionalen 
		Abh\"angigkeiten herleitbar sein.
		$$FD(R)\equiv FD(R_1)\cup\ldots\cup FD(R_n)$$
	\end{itemize}
\end{block}
\end{frame}

\begin{frame}[t]
\frametitle{\insertsection}
\framesubtitle{\insertsubsection}
\begin{block}{\textbf{Bemerkung}}
	\begin{itemize}
%		\item	Normalisierte Schemata nicht immer erw\"unscht! Stichworte: Performanz, Query-Komplexit\"at $\Rightarrow$ Denormalisierung
		\item Normalisierung basiert auf Methodik und Systematik der \textbf{Normalformen} ...\\[12pt]
		\item Normalformen dienen zur L\"osung der Anomalien- und Redundanz-Probleme
		\item Jede Normalform erf\"ullt bestimmte Eigenschaften, die einen Teil dieser Probleme l\"osen
	\end{itemize}
\end{block}
\end{frame}

\begin{frame}[t]
\frametitle{\insertsection}
\framesubtitle{\insertsubsection}
Arten von Normalformen:
\begin{itemize}
	\item \textbf{First Normal Form}\\[4pt]
	\item \textbf{Second Normal Form}\\[4pt]
	\item \textbf{Third Normal Form}\\[4pt]
	\item \textbf{Boyce-Codd Normal Form}\\[4pt]
	\item Fourth Normal Form\\[4pt]
	\item Fifth Normal Form (oder Project-Join Normal Form)\\[4pt]
	\item Inclusion Dependency Normal Form\\[4pt]
	\item Domain-Key Normal Form
\end{itemize}
\end{frame}

\subsection{Erste Normalform}

\begin{frame}[t]
\frametitle{\insertsection}
\framesubtitle{\insertsubsection}
\onslide
\begin{definition}[Erste Normalform (1NF)]\label{def:1nf}
	Ein Relationenschema $\mcl{R}$ ist in 1NF genau dann, wenn alle Attribute von $\mcl{R}$ ausschließlich atomare Werte 
	annehmen k\"onnen. Mehrwertige oder zusammengesetzte Attribute sind nicht gestattet.
\end{definition}
\pause
\nl\\[18pt]
{Definitionsgem\"a\ss} befindet sich jede Relation automatisch in 1NF.
\nl
Codd hat diese Normalform aber nochmals explizit formuliert.
\end{frame}

\begin{frame}[t]\frametitle{\insertsection}
\framesubtitle{\insertsubsection}
\structure{Die folgende Relation \texttt{Person} befindet sich \textit{nicht} in 1NF:}
\begin{center}
	\begin{tabular}{|c|c|c|c|}\hline
		\multicolumn{4}{|c|}{\small \textbf{\texttt{Person}}}\\\hline\hline
		\small \textbf{\key{\texttt{PNr}}} & \small \textbf{\texttt{Titel}}&\small \textbf{\texttt{Vorname}}&\small\textbf{\texttt{Nachname}} \\\hline 
		\small 1 &\small B.Sc. & \small Hans &\small Schneider  \\\hline 
		\small 2 &\small Dipl.-Kfm. & \small Max &\small Mustermann \\\hline 
		\small 3 &\cellcolor{Red}\small Dipl.-Inf. Dipl.-Kffr. &\small Erika &\small Musterfrau  \\\hline 
	\end{tabular}
\end{center}
In diesem Beispiel hat Frau Musterfrau zwei akademische Titel, die nicht unabh\"angig voneinander verarbeitet werden k\"onnen. 
\\[4pt]
Eine Operation auf dem Attribut \texttt{Titel} umfasst immer die \"Anderung beider Titel.
\end{frame}

\begin{frame}[t]\frametitle{\insertsection}
\framesubtitle{\insertsubsection}
\structure{Normalisierung: Erzeugung einer 1NF}
\begin{center}
\begin{tabular}{|c|c|c|}\hline
	\multicolumn{3}{|c|}{\small \textbf{\texttt{Person}}}\\\hline\hline
	\small \textbf{\key{\texttt{PNr}}} & \small \textbf{\texttt{Vorname}}&\small\textbf{\texttt{Nachname}} \\\hline 
	\small 1 & \small Hans &\small Schneider  \\\hline 
	\small 2 & \small Max &\small Mustermann \\\hline 
	\small 3 & \small Erika &\small Musterfrau  \\\hline 
\end{tabular}
\hspace{3mm}
\begin{tabular}{|c|c|}\hline
	\multicolumn{2}{|c|}{\small \textbf{\texttt{Titel}}}\\\hline\hline
	\small \textbf{\key{\texttt{PNr}}} & \small \textbf{\key{\texttt{Titel}}}\\\hline 
	\small 1 & \small B.Sc. \\\hline 
	\small 2 & \small Dipl.-Kfm.  \\\hline 
	\small 3 & \small Dipl.-Inf. \\\hline 
	\small 3 & \small Dipl.-Kffr. \\\hline 
\end{tabular}
\end{center}
\begin{itemize}
\item Aus dem mehrwertigen Attribut wird eigene Relation \texttt{Titel}.
\item Verkn\"upfung mit Relation \texttt{Person} \"uber deren Prim\"arschl\"ussel \texttt{PNr}.
\item	Prim\"arschl\"ussel von Relation \texttt{Titel}: Fremdschl\"ussel \texttt{PNr} und Attribut \texttt{Titel}.
\item Also: 1:n-Beziehung zwischen den Relationen \texttt{Person} und \texttt{Titel}.
\end{itemize}
\end{frame}

\begin{frame}[t]\frametitle{\insertsection}
\framesubtitle{\insertsubsection}
\structure{\textbf{Atomar oder nicht atomar?}}\\[4pt]
Ob ein Attribut atomar ist oder nicht, ist von der Verwendung des Attributs abh\"angig. 
\begin{block}{Beispiel}
	\begin{itemize}
		\item In einem Attribut \texttt{Vorname} wird f\"ur eine Person mit zwei Vornamen der Wert \textit{Hans Friedrich} gespeichert. 
		\texttt{Vorname} ist damit ein mehrwertiges Attribut. 
		\item Falls die Anwendung lediglich die Speicherung der Vornamen ben\"otigt, kann das Attribut als atomar betrachtet werden - 
		beide Vornamen werden wie \textit{ein} Vorname behandelt. 
		\item F\"ur eine Abfrage, wie viele Personen einen zweiten Vornamen besitzen, w\"are eine Zerlegung jedoch wieder sinnvoll.
	\end{itemize}
\end{block}
\abs
\pause
\alert{Es muss also vorab gepr\"uft werden, ob die Zerlegung eines mehrwertigen Attributs sinnvoll ist oder ob es als atomares 
	Attribut behandelt werden soll.}
\end{frame}

\begin{frame}[t]\frametitle{\insertsection}
\framesubtitle{\insertsubsection}
\structure{\textbf{1NF, Redundanzen, Anomalien}}\\[4pt]
Eine Relation in 1NF ist nicht notwendigerweise redundanz- oder anomaliefrei:
\begin{center}
	\begin{tabular}{|c|c|c|c|c|c|}\hline
		\multicolumn{6}{|c|}{\small \textbf{Mitarbeiter\_Projekt}}\\\hline\hline
		\small \textbf{\key{MANr}} & \small \textbf{Vorname}&\small \textbf{Nachname}&\textbf{\key{PNr}} &\small \textbf{Projektname}&\small \textbf{Funktion} \\\hline 
		\small 1 &\small Max & \small Mustermann &\small 4711 &\small Entwicklung &\small Entwickler \\\hline 
		\small 1 &\small \cellcolor{red}Max & \small \cellcolor{red}Mustermann &\small 4713 &\small Vertrieb & \small Kampagnen \\\hline 
		\small 2 &\small Erika &\small Musterfrau &\small 4711 &\small Entwicklung &\small SCRUM Master \\\hline 
		\small 2 &\small \cellcolor{red}Erika &\small \cellcolor{red}Musterfrau &\small 4712 &\small Test &\small Integrations-Tests \\\hline 
		\small 3 &\small Moritz & \small Musterherr &\small 4713 &\small Research &\small Leitung \\\hline 
		\small 4 &\small Karla & \small Musterdame &\small 4713 &\small \cellcolor{red}Vertrieb &\small \cellcolor{red}Kampagnen \\\hline 
	\end{tabular}
\end{center}
\end{frame}

\subsection{Zweite Normalform}

\begin{frame}[t]
\frametitle{\insertsection}
\framesubtitle{\insertsubsection}
\begin{block}{\textbf{Motivation der Zweiten Normalform}}
	\begin{itemize}
		\item Jeder Kandidatenschl\"ussel repr\"asentiert ein Konzept des Relationenschemas.
		\item Alle Attribute, die zu keinem Kandidatenschl\"ussel geh\"oren, sollen einen Fakt oder ein Merkmal 
		jedes Konzepts des Schemas darstellen.	
	\end{itemize}
\end{block}
\pause
\begin{definition}[Zweite Normalform (2NF)]\label{def:2nf}
	Ein Relationenschema $\mcl{R}$ ist in 2NF, wenn es in 1NF ist und jedes Attribut $A$ von $\mcl{R}$, das nicht zu einem  
	Kandidatenschlüssel geh\"ort, voll funktional von jedem Kandidatenschlüssel abhängt.
\end{definition}
\pause
\begin{block}{\textbf{Folge}}
	\begin{itemize}
		\item Relationenschema in 1NF, dessen Kandidatenschl\"ussel nur ein Attribut haben, ist in 2NF.
		\item Test auf 2NF muss daher nur f\"ur zusammengesetzte Kandidatenschl\"ussel erfolgen.
	\end{itemize}
\end{block}
\end{frame}

\begin{frame}[t]\frametitle{\insertsection}
\framesubtitle{\insertsubsection}
\onslide
\structure{Die folgende Relation ist nicht in 2NF:}
\begin{center}
	\begin{tabular}{|c|c|c|c|c|}\hline
		\multicolumn{5}{|c|}{\small \textbf{\texttt{Mitarbeiter-Projekt-Fkt}}}\\\hline\hline
		\small \textbf{\key{\texttt{MANr}}} & \small \textbf{\texttt{Vorname}}&\small \textbf{\texttt{Nachname}}
		   &\small\textbf{\key{\texttt{PNr}}} &\small \textbf{\texttt{Funktion}} \\\hline 
		\small 1 &\small Max & \small Mustermann &\small 4712 &\small Unit-Tests \\\hline 
		\small 1 &\small Max & \small Mustermann &\small 4713 & \small Kampagnen \\\hline 
		\small 2 &\small Erika &\small Musterfrau &\small 4712 &\small Integrations-Tests \\\hline 
	\end{tabular}
\end{center}
\hspace*{7em}Funktionale Abhängigkeiten:
\hspace*{7em}$\{\texttt{MANr},\texttt{PNr}\}\rightarrow\{\texttt{MANr},\texttt{PNr},\texttt{Vorname},\texttt{Nachname},\texttt{Funktion}\}$
\\
\hspace*{7em}$\{\texttt{MANr}\}\rightarrow\{\texttt{Vorname},\texttt{Nachname}\}$
\pause
\abs
\textbf{Begr\"undung:} \texttt{Vorname} und \texttt{Nachname} nicht voll funktional von Prim\"arschlüssel 
$\{\texttt{MANr},\texttt{PNr}\}$ (einziger Kandidatenschl\"ussel), sondern von Teilmenge $\{\texttt{MANr}\}$ 
dieses Schlüssels funktional abhängig.
\end{frame}

\begin{frame}[t]\frametitle{\insertsection}
	\framesubtitle{\insertsubsection}
	\structure{Zerlegung in 2NF:}
	 \begin{center}
		    \begin{tabular}{|c|c|c|}\hline
					\multicolumn{3}{|c|}{\small \textbf{\texttt{Person}}}\\\hline\hline
					 \small \textbf{\key{\texttt{MANr}}} & \small \textbf{\texttt{Vorname}}&\small\textbf{\texttt{Nachname}} \\\hline 
					\small 1 & \small Max &\small Mustermann  \\\hline 
					\small 2 & \small Erika &\small Musterfrau \\\hline 
				\end{tabular}
				\hspace{3mm}
				\begin{tabular}{|c|c|c|}\hline
			\multicolumn{3}{|c|}{\small \textbf{\texttt{Mitarbeiter-Projekt-Fkt}}}\\\hline\hline
			 \small \textbf{\key{\texttt{MANr}}} & \small\textbf{\key{\texttt{PNr}}} &\small \textbf{\texttt{Funktion}} \\\hline 
			\small 1 &\small 4712 &\small Unit-Tests \\\hline 
			\small 1 &\small 4713 & \small Kampagnen \\\hline 
			\small 2 &\small 4712 &\small Integrations-Tests \\\hline 
		\end{tabular}
			\end{center}
\begin{itemize}
	\item Funktionale Abh\"angigkeit $\{\texttt{MANr}\}\rightarrow\{\texttt{Vorname}, \texttt{Nachname}\}$ 
	in eigene Relation \texttt{Person} \"uberf\"uhren
	\item Determinante \texttt{MANr} wird Prim\"arschl\"ussel
	\item Verbindung zu 'Rest-Relation' \texttt{Mitarbeiter-Projekt-Fkt} \"uber Fremdschl\"usselbeziehung 
\end{itemize}
\end{frame}

\subsection{Dritte Normalform}

\begin{frame}[t]\frametitle{\insertsection}
\framesubtitle{\insertsubsection}
\textbf{Motivation der Dritten Normalform}
\\[4pt]
Durch 2NF k\"onnen einige Redundanzen und Anomalien beseitigt werden. 
\pause
\\[4pt]
\alert{Aber: Die folgende Relation befindet sich in 2NF und enth\"alt immer noch Redundanzen.}
\begin{center}
	\begin{tabular}{|c|c|c|c|c|}\hline
		\multicolumn{5}{|c|}{\small \textbf{\texttt{Mitarbeiter-Projekt-Fkt}}}\\\hline\hline
		\small \textbf{\key{\texttt{MANr}}} & \small \textbf{\texttt{Vorname}}&\small \textbf{\texttt{Nachname}}
		  &\small\textbf{\texttt{PLZ}} &\small \textbf{\texttt{Ort}} \\\hline 
		\small 1 &\small Max & \small Mustermann &\small 68165 &\cellcolor{Red}\small Mannheim \\\hline 
		\small 2 &\small Erika &\small Musterfrau &\small 57072 &\small Siegen \\\hline 
		\small 3 &\small Hans &\small Schmidt &\small 68165 &\cellcolor{Red}\small Mannheim \\\hline 
	\end{tabular}
\end{center}
\hspace*{9em}{Funktionale Abh\"angigkeiten:}\\
\hspace*{9em}$\{\texttt{MANr}\}\rightarrow\{\texttt{Vorname},\texttt{Nachname},\texttt{PLZ},\texttt{Ort}\}$
\\
\hspace*{9em}$\{\texttt{PLZ}\}\rightarrow\{\texttt{Ort}\}$
\pause
\\[4pt]
\textbf{Grund:} Funktionale Abh\"angigkeit eines Attributes von einem Nicht-Schl\"ussel.
\end{frame}

% Nicht verwendete alternative Definition (JR), jedoch ist hier Transitivit\"at nicht definiert!
\nowrite{
	\begin{frame}[t]\frametitle{\insertsection}
	\framesubtitle{\insertsubsection}
	\begin{definition}[Dritte Normalform (3NF)]
		\label{def:3nf}
		Eine Relation $R$ ist genau dann in 3NF, wenn sie in 2NF ist und kein Nicht-Schlüssel-Attribut transitiv von einem 
		Schlüsselkandidaten abhängt.
	\end{definition}
	Transitivitätsregel: $\{X\rightarrow Y\}\wedge \{Y\rightarrow Z\}\Longrightarrow\{X\rightarrow Z\}$
	\vspace{3mm}
	\alert{Für die 3NF darf es kein solches $Y$ unter den Nicht-Schlüssel-Attributen geben.}
\end{frame}
}

\begin{frame}[t]\frametitle{\insertsection}
\framesubtitle{\insertsubsection}
\begin{definition}[Dritte Normalform (3NF)]\label{def:3nf}
Ein Relationenschema $\mcl{R}(\mcl{A})$ ist in 3NF, wenn f\"ur jede funktionale Abh\"angigkeit $X\rightarrow A$ 
mit $X\subseteq\mcl{A}$ und $A\in\mcl{A}$ eine der folgenden Bedingungen gilt:
\\[4pt]
\begin{itemize}
	\item $X$ ist Superschl\"ussel von $\mcl{R}$.\\[4pt]
	\item $A\in X$, d.~h.~$X\rightarrow A$ ist trivial.\\[4pt]
	\item $A$ ist in einem Kandidatenschl\"ussel von $\mcl{R}$ enthalten.
\end{itemize}
\end{definition}
\pause
\abs
\begin{lemma}
Ist ein Relationenschema $\mcl{R}(\mcl{A})$ in 3NF, dann ist es auch in 2NF.
\end{lemma}
\end{frame}

\begin{frame}[t]\frametitle{\insertsection}
\framesubtitle{\insertsubsection}
\structure{Die folgende Relation befindet sich nicht in 3NF:}
\begin{center}
\begin{tabular}{|c|c|c|c|c|}\hline
\multicolumn{5}{|c|}{\small \textbf{\texttt{Mitarbeiter}}}\\\hline\hline
\small \textbf{\texttt{\key{MANr}}} & \small \textbf{\texttt{Vorname}}&\small \textbf{\texttt{Nachname}}
&\small\textbf{\texttt{PLZ}} &\small \textbf{\texttt{Ort}} \\\hline 
\small 1 &\small Max & \small Mustermann &\small 68165 &\small Mannheim \\\hline 
\small 2 &\small Erika &\small Musterfrau &\small 57072 &\small Siegen \\\hline 
\small 3 &\small Hans &\small Schmidt &\small 68165 &\small Mannheim \\\hline 
\end{tabular}
\end{center}
\textbf{Funktionale Abhängigkeiten:}
\nl
$\texttt{MANr}\rightarrow\{\texttt{Vorname}, \texttt{Nachname}, \texttt{PLZ}, \texttt{Ort}\}$
\\
$\texttt{PLZ}\rightarrow\texttt{Ort}$
\pause
\\[8pt]
\textbf{Begr\"undung:} 
\nl
$\texttt{PLZ}$ ist nicht Superschl\"ussel, $\texttt{Ort}\notin\{\texttt{PLZ}\}$, $\texttt{Ort}$ geh\"ort nicht zu einem Kandidatenschl\"ussel.
\end{frame}

\begin{frame}[t]\frametitle{\insertsection}
\framesubtitle{\insertsubsection}
\structure{Zerlegung in 3NF:}
\begin{center}
\begin{tabular}{|c|c|c|c|}\hline
\multicolumn{4}{|c|}{\small \textbf{\texttt{Mitarbeiter}}}\\\hline\hline
\small \textbf{\texttt{\key{MANr}}} & \small \textbf{\texttt{Vorname}}&\small \textbf{\texttt{Nachname}}
&\small\textbf{\texttt{PLZ}}  \\\hline 
\small 1 &\small Max & \small Mustermann &\small 68165 \\\hline 
\small 2 &\small Erika &\small Musterfrau &\small 57072\\\hline 
\small 3 &\small Hans &\small Schmidt &\small 68165 \\\hline 
\end{tabular}
\hspace{3mm}
\begin{tabular}{|c|c|}\hline
\multicolumn{2}{|c|}{\small \textbf{\texttt{Ort}}}\\\hline\hline
\small\textbf{\texttt{\key{PLZ}}} & \small\textbf{\texttt{Ort}}\\\hline
\small 68165 & \small Mannheim\\\hline 
\small 57072 & \small Siegen \\\hline
\end{tabular}
\end{center}
\pause
Die 'schlechte' funktionale Abh\"angigkeit $\texttt{PLZ}\rightarrow\texttt{Ort}$ wird in eine eigene Relation \"uberf\"uhrt: 
\begin{itemize}
\item Determinante \texttt{PLZ} wird zum Prim\"arschl\"ussel
\item \texttt{PLZ} über Fremdschl\"usselbeziehung mit Mitarbeiter-Relation verkn\"upft
\end{itemize}
\end{frame}

\begin{frame}[t]\frametitle{\insertsection}
\framesubtitle{\insertsubsection}
\structure{Allgemein gilt:}
\\[8pt]
Ein Relationenschema $\mcl{R}$ kann stets durch den \emph{Synthesealgorithmus} verlustlos und abh\"angigkeitserhaltend in 3NF-Schemata 
$\mcl{R}_1,\ldots,\mcl{R}_n$ zerlegt werden.
\pause
\\[36pt]
Vorteil der 3NF:
\\[8pt]
Beseitigung einiger Redundanzen durch Entfernung von Nicht-Schl\"ussel-Abhängigkeiten.
\end{frame}

\begin{frame}[t]\frametitle{\insertsection}
\framesubtitle{\insertsubsection}
In welcher NF befindet sich die folgende Relation?
\abs
%
% Antwort: in 3NF
%
\begin{center}
\begin{tabular}{|c|c|c|}\hline
\multicolumn{3}{|c|}{\small \textbf{\texttt{MITARBEITER-PROJEKT-GRUPPENLEITUNG}}}\\\hline\hline
\small \textbf{\key{MANr}} & \small \textbf{\texttt{\key{PROJEKT}}}&\small \textbf{\texttt{PROJEKTLEITER}} \\\hline 
\small 1 &\small Entwicklung A& \small Meier  \\\hline 
\small 2 &\small Entwicklung A &\small Schulz \\\hline 
\small 2 &\small Entwicklung B &\small Schmidt \\\hline 
\small 3 &\small Entwicklung B &\small Schmidt \\\hline 
\end{tabular}
\end{center}
\end{frame}

\subsection{Boyce-Codd Normalform}

% Nicht verwendete alternative Definition (JR), jedoch ist hier Transitivit\"at nicht definiert!
\nowrite{
\begin{frame}[t]\frametitle{\insertsection}
\framesubtitle{\insertsubsection}
\begin{definition}[Boyce-Codd Normalform (BCNF)]
\label{def:bcnf}
Eine Relation $R$ ist genau dann in BCNF, wenn sie in 3NF ist und kein Attribut transitiv von einem Schlüsselkandidaten abhängt.
\end{definition}
\vspace{3mm}
\alert{Verallgemeinerung der 3NF: die BCNF bezieht sich auf alle Attribute, nicht nur auf die Nicht-Schlüssel-Attribute. Sie ist daher eine 
Verschärfung der 3NF: Die BCNF verhindert, dass Teile von Schlüsselkandidaten funktionale Abhängigkeiten aufweisen.}
\end{frame}
}

\begin{frame}[t]\frametitle{\insertsection}
\framesubtitle{\insertsubsection}
\begin{definition}[Boyce-Codd-Normalform (BCNF)]\label{def:bcnf}
Ein Relationenschema $\mcl{R}(\mcl{A})$ ist in BCNF, wenn f\"ur jede funktionale Abh\"angigkeit $X\rightarrow Y$ 
mit $X,Y\subseteq\mcl{A}$ eine der folgenden Bedingungen gilt:
\\[4pt]
\begin{itemize}
\item $X$ ist Superschl\"ussel von $\mcl{R}$.\\[4pt]
\item $Y\subseteq X$, d.~h.~$X\rightarrow Y$ ist trivial.
\end{itemize}
\end{definition}
\pause
\abs
\begin{lemma}
Ist ein Relationenschema $\mcl{R}(\mcl{A})$ in BCNF, dann ist es auch in 3NF.
\end{lemma}
\end{frame}

\begin{frame}[t]
\frametitle{\insertsection}
\framesubtitle{\insertsubsection}
\structure{Die folgende Relation befindet sich nicht in BCNF:}
\begin{center}
\begin{tabular}{|c|c|c|}\hline
\multicolumn{3}{|c|}{\small \textbf{\texttt{MITARBEITER-PROJEKT-GRUPPENLEITUNG}}}\\\hline\hline
\small \textbf{\key{\texttt{MANr}}} & \small \textbf{\key{\texttt{PROJEKT}}}&\small \textbf{\texttt{PROJEKTLEITER}} \\\hline 
\small 1 &\small Entwicklung A& \small Meier  \\\hline 
\small 2 &\small Entwicklung A &\small Schulz \\\hline 
\small 2 &\small Entwicklung B &\small Schmidt \\\hline 
\small 3 &\small Entwicklung B &\small Schmidt \\\hline 
\end{tabular}
\end{center}
Funktionale Abh\"angigkeiten: 
% Sachverhalt:
% > Mitarbeiter arbeiten in Projekten.
% > Ein Mitarbeiter kann in verschiedenen Projekten arbeiten.
% > Ein Mitarbeiter kann in h¨ochstens einem Projekt als Projektleiter auftreten.
\begin{itemize}
\item $\{\texttt{MANr}, \texttt{PROJEKT}\}\rightarrow\texttt{PROJEKTLEITER}$ (Prim\"arschl\"ussel)
%\item $\{\texttt{MANr}, \texttt{PROJEKTLEITER}\}\rightarrow\texttt{PROJEKT}$ (Kandidatenschl\"ussel)
\item $\texttt{PROJEKTLEITER}\rightarrow \texttt{PROJEKT}$
\end{itemize}
\pause\abs
\textbf{Begr\"undung:}
\nl
\texttt{PROJEKTLEITER} ist nicht Superschl\"ussel, $\texttt{PROJEKTLEITER}\rightarrow \texttt{PROJEKT}$ nicht trivial
\end{frame}

\begin{frame}[t]\frametitle{\insertsection}
\framesubtitle{\insertsubsection}
\structure{Zerlegung in BCNF:}
\begin{center}
\begin{tabular}{|c|c|}\hline
\multicolumn{2}{|c|}{\small \textbf{\texttt{MITARB-PROJ-GRP}}}\\\hline\hline
\small \textbf{\key{\texttt{MANr}}} & \small \textbf{\key{\texttt{PROJEKTLEITER}}} \\\hline 
\small 1 & \small Meier  \\\hline 
\small 2 & \small Schulz \\\hline 
\small 2 & \small Schmidt \\\hline 
\small 3 & \small Schmidt \\\hline 
\end{tabular}
\hspace{3mm}
\begin{tabular}{|c|c|}\hline
\multicolumn{2}{|c|}{\small \textbf{\texttt{PROJ-GRP-PROJEKT}}}\\\hline\hline
\small \textbf{\key{\texttt{PROJEKTLEITER}}} & \small \textbf{\texttt{PROJEKT}} \\\hline 
\small Meier & \small Entwicklung A  \\\hline 
\small Schulz & \small Entwicklung B \\\hline 
\small Schmidt & \small Entwicklung B \\\hline 
\end{tabular}
\end{center}
'Schlechte' funktionale Abh\"angigkeit $\texttt{PROJEKTLEITER}\rightarrow \texttt{PROJEKT}$ in eigene Relation auslagern: 
\begin{itemize}
\item Determinante \texttt{PROJEKTLEITER} wird zum Prim\"arschl\"ussel in dieser Relation.
\item \texttt{PROJEKTLEITER} wird zum Fremdschl\"ussel in der \texttt{MITARB-PROJ-GRP}-Relation. 
% (JR):
%	Die Determinante eines Teilschlüssels wird in eine neue Relation ausgelagert. Die Determinante
%	wird zum neuen Primärschlüssel. In der Relation wird der abhängige Teilschlüssel ergänzt.
\end{itemize}
\pause
\abs
\alert{Prim\"arschl\"ussel $\{\texttt{MANr}, \texttt{PROJEKT}\}\rightarrow\texttt{PROJEKTLEITER}$ geht bei Zerlegung verloren.}
\end{frame}

\begin{frame}[t]\frametitle{\insertsection}
\framesubtitle{\insertsubsection}
\structure{Allgemein gilt:}
\\[8pt]
Ein Relationenschema $\mcl{R}$ kann stets durch den \emph{Dekompositionsalgorithmus} verlustlos in BCNF-Schemata 
$\mcl{R}_1,\ldots,\mcl{R}_n$ zerlegt werden.
\pause
\\[36pt]
\alert{Diese Zerlegung ist nicht immer abh\"angigkeitserhaltend.}
\end{frame}

\nowrite{
\section*{Übungsaufgaben}
%\subsection*{Grundlegendes Datenbankverständnis}
\begin{frame}{\insertsection}
	%\framesubtitle{\insertsubsection}
	\begin{alertblock}{Normalisierung}
		{
			Gegeben ist die folgende Relation: 			
			\centering			
			\small
			\begin{tabular}{|c|c|c|c|}\hline
				\multicolumn{4}{|c|}{\textbf{Buchhandel}}\\\hline
				\textbf{InvNr} & \textbf{Titel} & \textbf{ISBN} & \textbf{Autor}\\\hline\hline
				1 & Per Anhalter durch die Galaxis & 0-4561-123 & Adams \\\hline 
				2 & Datenbank-Systeme & 3-1234-567 & Kemper und Eickler \\\hline 
				3 & Datenbank-Systeme & 4-5678-901 & Elmasri und Navathe \\\hline 
				4 & Machs gut und danke für den Fisch& 2-3456-789 & Adams\\\hline 
			\end{tabular}			
		}
				\begin{itemize}
					\item In welcher Normalform befindet sich die unten stehende Relation? 
					\item Überführen Sie die Relation in die 3NF.
				\end{itemize}		
	\end{alertblock}
\end{frame}
%
\begin{frame}{\insertsection}
	%\framesubtitle{\insertsubsection}
	\begin{alertblock}{Normalisierung}
		{
			Gegeben ist die folgende Relation: 			
			\centering			
			\small
			\begin{tabular}{|c|c|c|c|c|c|c|}\hline
				\multicolumn{7}{|c|}{\textbf{Ergebnisse}}\\\hline
				\textbf{MatrNr} & \textbf{Name} & \textbf{Stgang} & \textbf{StRichtung} & \textbf{Fach} &  \textbf{Semester} & \textbf{Note} \\\hline\hline
				4711 & Mustermann & WI & AM &  DB 1 & 3 & 1.4 \\\hline
				4713 & Schulz & WI & SE &  DB 1 & 3 & 1.6 \\\hline
				4713 & Schulz & WI & SE &  VertSys & 4 & 1.8 \\\hline
				4712 & Musterfrau & WI & AM &  DB 1 & 3 & 1.2 \\\hline
				4715 & Schröder & WI & SE &  DB 1 & 3 & 5.0 \\\hline
				4711 & Mustermann & WI & AM &  DB 2 & 4 & 2.0 \\\hline
				4712 & Musterfrau & WI & AM &  DB 2 & 4 & 2.2 \\\hline
				4715 & Schröder & WI & SE &  VertSys & 4 & 2.2 \\\hline
			\end{tabular}			
		}
		\begin{itemize}
			\item In welcher Normalform befindet sich die unten stehende Relation? 
			\item Überführen Sie die Relation in die 3NF.
		\end{itemize}
	\end{alertblock}	
\end{frame}
}
