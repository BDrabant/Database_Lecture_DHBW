%!TEX root = Slides.tex
\section{Anwendungsf\"alle -- Programmierung}

\begin{frame}[t]
\frametitle{\insertsection}
\structure{\textbf{Auftragsmanagement}}\\[4pt]
\structure{\textbf{Universit\"atsverwaltung}}\\[8pt]
\begin{itemize}
	\item Programmierung von Java-Applikationen f\"ur die (parametrisierbare) Abfrage
	\item Pr\"asentation der Ergebnisse durch die Teams
\end{itemize}
\end{frame}

\subsection*{Auftragsmanagement}

\begin{frame}[t]
\frametitle{\insertsection}
\framesubtitle{\insertsubsection}
\begin{itemize}
\item Implementieren Sie die Datenbankabfragen aus den Aufgaben \ref{A1} -- \ref{A8} des Auftragsmanagements in Java. 
\item Parametrisieren Sie die in Java implementierten Abfragen aus den Aufgaben \ref{A4}, \ref{A7}, \ref{A8}. 
\end{itemize}
\end{frame}

\subsection*{Universit\"atsverwaltung}

\begin{frame}[t]
\frametitle{\insertsection}
\framesubtitle{\insertsubsection}
\begin{itemize}
	\item Implementieren Sie die Datenbankabfragen aus den Aufgaben \ref{U1}, \ref{U2}, \ref{U3}, \ref{U4} der 
	Universit\"atsverwaltung in Java. 
	\item Parametrisieren Sie wenn m\"oglich die in Java implementierten Abfragen. 
	\end{itemize}
\end{frame}
