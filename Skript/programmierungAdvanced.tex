%!TEX root = Slides.tex
\part{Fortgeschrittene Datenbankprogrammierung}
\label{part:programmierungAdvanced}

\section{JDBC}
\subsection{Connection Pooling}

\begin{frame}[t]\frametitle{\insertsection}
\framesubtitle{\insertsubsection}
\structure{Verbindungsauf- und -abbau ist Flaschenhals bei Datenbankprogrammierung}
\begin{itemize}
	\item Verbindungsaufbau kostet im Allgemeinen sehr viel Zeit 
	\item Auf einzelner Verbindung kann nicht parallel gearbeitet werden. Mehrere Threads müssen 
	eigene Verbindungsobjekte erzeugen.
\end{itemize}
\onslide\pause 
\begin{block}{Connection Pooling}
	\begin{itemize}
		\item Mehre Verbindungen werden beim Start der Anwendung aufgebaut und in Pool gehalten
		\item Verbindungen aus Pool können sofort verwendet werden
		\item Nicht mehr benötigte Verbindungen werden nicht abgebaut, sondern in Pool zurückgegeben 
		\item Parameter steuern Verhalten des Pools -- wie viele Verbindungen usw.
	\end{itemize}
\end{block}
\end{frame}

\begin{frame}[fragile]\frametitle{\insertsection}
\framesubtitle{\insertsubsection}
\structure{Connection Pooling muss nicht selbst implementiert werden}
\abs
Verwende Bibliotheken f\"ur JDBC-konformes Connection Pooling. 
\nl
Beispiel: Open Source Bibliothek C3PO.
\abs
\onslide\pause
\begin{itemize}
\item C3PO verf\"ugbar unter \url{http://sourceforge.net/projects/c3p0/}
\item Paket muss im \texttt{CLASSPATH} liegen und kann sofort verwendet werden
\item Verbindungen zu Datenbanken werden über eigene Datenquelle (Data Source) bereitgestellt
\end{itemize} 
\end{frame}

\begin{frame}[fragile]\frametitle{\insertsection}
\framesubtitle{\insertsubsection\ -- Code-Beispiel}
\begin{lstlisting}[xleftmargin=3ex, showstringspaces=false, language=Java]
//Erstellen einer Datenquelle
ComboPooledDataSource cpds = new ComboPooledDataSource(); 
cpds.setDriverClass("org.postgresql.Driver");
cpds.setJdbcUrl("jdbc:postgresql://<host><:port></database>");
cpds.setUser("db");
cpds.setPassword("db"); 
cpds.setMinPoolSize(5); 
cpds.setAcquireIncrement(5);
cpds.setMaxPoolSize(100); 

//Verwendung der Datenquelle
Connection con1 = cpds.getConnection();
Connection con2 = cpds.getConnection();
...
\end{lstlisting}
\end{frame}

\begin{frame}[fragile]\frametitle{\insertsection}
\framesubtitle{\insertsubsection\ -- Web App Server}
\structure{Die meisten Web Application Server verf"ugen über eigenes Connection Pooling mittels JNDI}
\vspace{3mm}
Beispiel Apache Tomcat: Deskriptor \texttt{server.xml} und Java-Code gegenübergestellt
\begin{columns}
  \begin{column}{.48\textwidth}
  	\lstset{language=xml}
		\lstset{escapeinside={(*@}{@*)}}
\begin{lstlisting}[xleftmargin=3ex, basicstyle=\ttfamily\tiny, showstringspaces=false]
<Context>
 <Resource name="jdbc/TestDB" 
  auth="Container" type="javax.sql.DataSource" 
  maxActive="100" 
  maxIdle="30" 
  maxWait="1000" 
  username="<user>" 
  password="<password>" 
  driverClassName="org.postgresql.Driver"
  url="jdbc:postgresql://<host><:port></database>"/> 
</Context>
\end{lstlisting}
	\end{column}
	\begin{column}{.48\textwidth}
\lstset{language=Java}
			\lstset{escapeinside={(*@}{@*)}}
\begin{lstlisting}[xleftmargin=3ex, basicstyle=\ttfamily\tiny, showstringspaces=false, language=Java]
//Initialisierung
Context initContext = new InitialContext(); 
Context envContext = 
  (Context)initContext.lookup("java:/comp/env");
DataSource ds = 
  (DataSource)envContext.lookup("jdbc/TestDB");
			
//Jetzt kann Verbindung angefordert werden
Connection conn = ds.getConnection(); 
\end{lstlisting}
  \end{column}
\end{columns}
Web App Server verwaltet eigenen Connection Pool und stellt ihn Java-Anwendungen zur Verf\"ugung
\end{frame}

\subsection{Entwicklung von Anwendungen mit Datenbankanbindung}

\begin{frame}
	\frametitle{\insertsection}
	\framesubtitle{\insertsubsection}
	\structure{Wie es \textbf{nicht} geht:}
	\begin{itemize}
		\item Datenbankverbindungen und -abfragen im Code verstreuen
		\item Immer wieder durch ResultSets iterieren und quasi tabellarisch arbeiten
	\end{itemize}
\abs
	\structure{Stattdessen:}
	\begin{itemize}
		\item Entwerfen einer geeigneten Softwarearchitektur
		\item Nutzung der M\"oglichkeiten der Objektorientierung
	\end{itemize}	
\end{frame}

\begin{frame}
	\frametitle{\insertsection}
	\framesubtitle{\insertsubsection}
	\onslide
	\structure{Systematischer Aufbau:}
	\begin{itemize}
		\item Entwurf von Containerklassen f\"ur die Entit\"aten und Relationenschemata
		\begin{itemize}
			\item Verwendung von getter- und setter-Methoden
		\end{itemize}
 	  \pause
		\item Entwurf einer Zugriffsschnittstelle, die auf Containerklassen zugreift
		\pause
		\item Implementierung der Zugriffsschnittstelle f\"ur die konkrete Datenbank
		\pause
		\item Zugriff auf die Daten im Code \textit{ausschließlich über Zugriffsschnittstelle}
		\begin{itemize}
			\item ... am besten mittels Generator- oder Factory-Klasse
		\end{itemize}
	\end{itemize}
\end{frame}

\begin{frame}[fragile]\frametitle{\insertsection}
	\framesubtitle{\insertsubsection\ -- Beispiel}
	Containerklasse f\"ur Relationenschema
	\nl
	\texttt{\small Auto(int ID, varchar Name, varchar Marke, int[3] Farbe, int Leistung)}:
\begin{lstlisting}[xleftmargin=3ex, language=Java]
public class Auto{
 private int id; 		
 private int[] farbe; 		
 private int power; 
 ...

 public Auto() {farbe = new int[3];}

 public void setID(int id) {this.id = id;}
 public int getID() {return this.id;}

 public void setFarbe(int[] farbe) {this.farbe = farbe;}
 public int[] getFarbe() {return this.farbe;}

 public void setLeistung(int leistung) {this.power = leistung;}
 public int getLeistung() {return this.power;}
 ...
}
\end{lstlisting}
\end{frame}

\begin{frame}[fragile]\frametitle{\insertsection}
	\framesubtitle{\insertsubsection\ -- Beispiel}
	\structure{Zugriffsinterface}
\begin{lstlisting}[xleftmargin=3ex, language=Java]
public interface IRepository{
  public Auto createAuto(); 
		
  public Auto[] lookupAutoByFarbe(char[] farbe); 
  		
  public Auto[] lookupAutoByLeisung(int leistung); 

  public Auto lookupAutoById(int id); 
		
  public void saveAuto(Auto a); 	
}
\end{lstlisting}
\end{frame}

\begin{frame}[fragile]\frametitle{\insertsection}
	\framesubtitle{\insertsubsection\ -- Beispiel}
	\structure{Implementierungsklasse von Zugriffsinterface f\"ur PostgreSQL:}
\begin{lstlisting}[xleftmargin=3ex, showstringspaces=false, language=Java]
public class PostgreSQLRepository implements IRepository{
  private Connection con; 
  ... 
  
  public Auto lookupAutoById(int id){
    PreparedStatement pst = new PreparedStatement("SELECT * from Auto WHERE id = ?"); 
    pst.setInt(1,id); 
    ResultSet rs = pst.execute(); 
    Auto a = new Auto(); 
    
    while (rs.next()){
      a.setId(rs.getInt(0));
      a.setFarbe(rs.getFarbe(3));
      ...
    }
    return a;
  }
}
\end{lstlisting}
\end{frame}

\begin{frame}[fragile]\frametitle{\insertsection}
\framesubtitle{\insertsubsection\ -- Beispiel}
\structure{Factory-Klasse zur Erzeugung von Zugriffsinterfaces}
\begin{lstlisting}[xleftmargin=3ex, showstringspaces=false, language=Java]
public class RepoFactory{
  ...
  public static IRepository createRepository(ConfigObject config){
    if (config.getDatabase() == "Postgresql"){
      ...
      return new PostgreSQLRepository(); 
    } 
    else if (config.getDatabase() == "MySQL"){
      ...
      return new MySQLRepository(); 
    }
  }
}
\end{lstlisting}
Annahme: Config-Objekt \texttt{config} Konfigurationsdatei, in der die Datenbank konfiguriert wird.
\end{frame}

\begin{frame}[fragile]\frametitle{\insertsection}
\framesubtitle{\insertsubsection}
\structure{Zusammenfassung: Verwendung der Klassen}
\begin{lstlisting}[xleftmargin=3ex, showstringspaces=false, language=Java]
public static void main(String args[]){
  
  ConfigObject config = ...
  
  IRepository repo = RepoFactory.createRepository(config); 
  Auto auto = new Auto();
  auto.setLeistung(120);
  repo.saveAuto(auto)
  repo.lookupAutoByLeistung(120);
  
  Auto nocheinauto = new Auto(); 
  nocheinauto.setFarbe("schwarz");
  repo.saveAuto(nocheinauto);   
}
\end{lstlisting}
\end{frame}

\section{Objekt-relationales Mapping}
\subsection{Begriffserklärung}

\begin{frame}[t]\frametitle{Objekt-relationales Mapping}
    \framesubtitle{Begriffserklärung}
    \structure{Objektorientierte Welt: Objekte haben starke Strukturierung}
    \begin{itemize}
    	\item Schachtelung und komplexe Abhängigkeiten von Objekten
    	\item Vererbungshierarchien 
    	\item Schnittstellenkonzepte
    	\item Polymorphie und abstrakte Klassen
    \end{itemize}
\end{frame}

\begin{frame}[t]\frametitle{Objekt-relationales Mapping}
\framesubtitle{Begriffserklärung}
\structure{Relationale Welt:}
\begin{itemize}
	\item Tabellen sind "flach"
	\item Abhängigkeiten \"uber referentielle Integrit\"at 
	\item Keine inher\"anten Vererbungs-, Methoden- oder Schnittstellenkonzepte
\end{itemize}
\end{frame}

\begin{frame}[t]\frametitle{Objekt-relationales Mapping}
\framesubtitle{Begriffserklärung}
\onslide
\structure{Br\"ucke zwischen den Welten}
\begin{itemize}
	\item Versuch des Mapping zwischen objektorientierter und relationaler Welt 
	\item Mittels \textit{Object Flattening} werden strukturierte Objekte auf flache Tabellen und Schemata abgebildet
	\item Object Flattening erlaubt nur in gewissem Umfang die R\"uckabbildung von Schemata in Objekte
\end{itemize}
\pause
\begin{definition}[Object-Relational Impedance Mismatch]
	Unter dem Begriff \textit{Object-Relational Impedance Mismatch} wird die grundlegende Unverträglichkeit 
	von objektorientierten Programmiersprachen mit den Konzepten relationaler Datenbanken verstanden. 
	Es besteht ein \textit{Paradigmenkonflikt}.	
\end{definition}
\end{frame}

\againframe<2>{isa}

\begin{frame}[t]\frametitle{Objekt-relationales Mapping}
\framesubtitle{Ansätze für die Abbildung von is-a -- Beziehungen}
\structure{Es wurden verschiedene Möglichkeiten diskutiert:}
\onslide
\begin{itemize}
	\item One Table per Class Hierarchy 
	\begin{itemize}
		\item Attribute aller Entit\"aten der Klassenhierarchie in gemeinsames Relationenschema. 
		\item Unterscheidung der Entit\"aten durch Diskriminator-Attribut.
		\item \textbf{Problem:} Ansatz einfach zu implementieren, provoziert jedoch \texttt{null}-Werte und wird f\"ur 
		komplexere Hierarchien schwer wartbar.\\[4pt]
		%\item Normalisierung? 
	\end{itemize}
	\pause
	\item One Table per Concrete Class 
	\begin{itemize}
		\item Alle konkreten Klassen in eigenem Relationenschema.
		\item \textbf{Problem:} Informationen der Vererbung gehen verloren. Abstrakte Klasse existiert nicht mehr im Datenmodell.\\[4pt]
	\end{itemize}
	\pause
	\item One Table per Class
	\begin{itemize}
		\item Alle Klassen (auch die abstrakten) werden in eigenem Relationenschema abgebildet.
		\item \textbf{Problem:} Alle Vererbungsinformationen vorhanden, aber erhöhter Join-Aufwand 						
	\end{itemize}
\end{itemize}
\end{frame}

\begin{frame}[t]\frametitle{Objekt-relationales Mapping}
\framesubtitle{Verschiedene Konzepte}
\alert{Fazit: O-R-Mapping ist grundsätzliches Abbildungsproblem. Es gibt keine einheitliche Lösung.}
\abs
Entwickler arbeiten auf Objekten und m\"ussen Speicherung der Daten und Zerlegung der Objekte in Relationen manuell durchführen.
\abs
Dazu gibt es verschiedene Ansätze:
\onslide\pause
\begin{itemize}
	\item Objektorientierte Datenbanken 
	\begin{itemize}
		\item Objekte k\"onnen direkt in Datenbank abgelegt werden
		\item Nicht standardisiert. Auslesen der Objekte nur mit Hilfe der Anwendung möglich
		\item F\"ur spezielle Anwendungsfälle
	\end{itemize}
	\pause
	\item Erweiterung objektorientierter Programmiersprachen um relationale Konzepte
	\begin{itemize}
		\item SQL direkt im Code eingebettet
		\item Lösung widerspricht in großen Teilen objektorientierten Ansätzen
		\item Wird in der heutigen Entwicklung fast nicht mehr eingesetzt
	\end{itemize}
	\pause
	\item O/R-Mapper ...
\end{itemize}
\end{frame}

\subsection{O/R-Mapper}

\begin{frame}[t]\frametitle{Objekt-relationales Mapping}
	\framesubtitle{O/R-Mapper Allgemein}
	\structure{O/R-Mapper bilden eine Zwischenschicht zwischen Anwendung und Datenbank}
  \onslide
	\begin{itemize}
		\item Sie erledigen das Mapping von Objekten auf Relationen 
		\item Transparent für Anwendungsentwicklung
		\item Weite Verbreitung\\[8pt]
		\pause
		\item Viele unterschiedliche Implementierungen und Standards 
		\begin{itemize}
			\item Microsoft .NET Entity Framework 
			\begin{itemize}
				\item Alle .NET-Programmiersprachen (C\#, Visual Basic, Visual C++)
				\item Gute Unterstützung verschiedener Datenbanken
				\item Unterstützt Code-First und Model-First Ansätze
			\end{itemize}
			\item Java Persistence API (JPA)
			\begin{itemize}
				\item Standardisierte O/R-Mapper-Spezifikation für Java
			\end{itemize}
		\end{itemize}
	\end{itemize}
\end{frame}

\begin{frame}[t]\frametitle{Objekt-relationales Mapping}
	\framesubtitle{Java Persistence API (JPA)}
	\onslide
	\structure{Die JPA wurde erstmals 2006 veröffentlicht}
	\abs
	\begin{itemize}
		\item Standardisierte Programmierschnittstelle (API)
		\begin{itemize}
			\item Methoden zum Speichern, Modifizieren und Laden von Objekten aus relationalen Datenbanken
			\item Unterstützung von Polymorphie / Vererbung 
			\item Verwendet Java Annotations (z.B. \texttt{@Entity, @OneToMany})
		\end{itemize}
	\pause
		\item Java Persistence Query Language 
		\begin{itemize}
			\item Eigene, an SQL angelehnte Sprache
		\end{itemize}
		\item XML-Datei zur Konfiguration des relationalen Mappings
		\begin{itemize}
			\item Mittels \texttt{persistence.xml}-Datei kann das Mapping definiert werden
		\end{itemize}
	\pause
		\item Viele Implementierungen der JPA, z.B. \textit{Eclipselink} oder \textit{Hibernate}
	\end{itemize}
\end{frame}

\begin{frame}[t]\frametitle{Objekt-relationales Mapping}
	\framesubtitle{EclipseLink}
	\structure{JPA-Referenzimplementierung \textit{EclipseLink}}
	\abs
	\begin{itemize}
		\item Kann unter \url{https://www.eclipse.org/eclipselink/} heruntergeladen werden
		\item Umfasst mehr als nur eine JPA-Implementierung
		\item Konfiguration über \texttt{persistence.xml}
		\item Verwendung von Annotationen im Code
		\item Muss im \texttt{CLASSPATH} vorhanden sein
	\end{itemize}
\end{frame}

\begin{frame}[fragile]\frametitle{Objekt-relationales Mapping}
    \framesubtitle{Beispiel -- Tabellen f\"ur Genre und Film}
    \structure{persistence.xml}
\lstset{language=xml}
\lstset{escapeinside={(*@}{@*)}}
\begin{lstlisting}[xleftmargin=3ex, language=xml]
<?xml version="1.0" encoding="UTF-8" ?>
<persistence xmlns:xsi=...> 
 <persistence-unit name="filme" transaction-type="RESOURCE_LOCAL">   
  <class>de.test.Film</class>   
  <class>de.test.Genre</class>   
  <properties>     
   <property name="javax.persistence.jdbc.driver" value="org.postgresql.Driver"/>     
   <property name="javax.persistence.jdbc.url" 
             value="jdbc:postgresql://<host><:port></database>"/>
   <property name="javax.persistence.jdbc.user" value="<user>"/>
   <property name="javax.persistence.jdbc.password" value="<password>"/>
   <property name="eclipselink.ddl-generation" value="create-tables"/>
   <property name="eclipselink.ddl-generation.output-mode" value="database"/>
  </properties>
 </persistence-unit>
</persistence> 
	\end{lstlisting}
\end{frame}

\begin{frame}[fragile]\frametitle{Objekt-relationales Mapping}
    \framesubtitle{Beispiel -- Tabellen f\"ur Genre und Film}
    \structure{Java-Code \texttt{\small Film(FilmNr int,Name varchar(50),ErschJahr int,GenreID int)}}
\lstset{language=Java}
\lstset{escapeinside={(*@}{@*)}}
\begin{lstlisting}[xleftmargin=3ex, basicstyle=\ttfamily\tiny, language=Java]
package de.test;
import javax.persistence.*;

@Entity
public class Film{
 @Id
 @GeneratedValue(strategy=GenerationType.IDENTITY)
 @Column(name="FilmNr")
 private long filmid ; 

 @Column(name="Name")
 private String name ; 

 @Column(name="ErschJahr")
 private int jahr;

 @Column(name="GenreID")
 private int genreid;

 @OneToOne(cascade=CascadeType.PERSIST)
 @JoinColumn(name="GenreID")
 private Genre genre ; 

 //Getter und Setter fuer private Attribute
 ...
}
\end{lstlisting}
\end{frame}

% genauer im code:
% import javax.persistence.CascadeType;
% import javax.persistence.Entity;
% import javax.persistence.GeneratedValue;
% import javax.persistence.GenerationType;
% import javax.persistence.Id;
% import javax.persistence.OneToOne;

%\section*{Zusammenfassung}
%
%\begin{frame}{Zusammenfassung}
%	\begin{itemize}
%		\item Grundlagen über Systemarchitekturen
%		\begin{itemize}
%					\item 2-Tier
%					\item 3-Tier
%		\end{itemize}		
%		\item JDBC
%		\begin{itemize}
%			\item Einführung
%			\item API-Struktur
%			\item Connection Pooling
%		\end{itemize}
%		\item Object-relationaler Impedance Mismatch und O/R - Mapper
%	\end{itemize}
%
%\end{frame}
%